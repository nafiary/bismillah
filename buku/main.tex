\documentclass[12pt,oneside,reqno]{ta-its}
\usepackage{hyperref}
\usepackage{listings}
\usepackage{float}
\usepackage{mdframed}
\usepackage{wrapfig}

\renewcommand{\lstlistingname}{Kode Sumber}
\renewcommand{\lstlistlistingname}{DAFTAR KODE SUMBER}

\title{Implementasi \textit{Publish/Subscribe} pada Rancang Bangun Sistem Monitoring Perangkat Jaringan di ITS}{Implementation of Publish/Subscribe on Design of Network Monitoring Device System in ITS}{KI141502}

\author{Afif Ridho Kamal Putra}{05111440000173}

\supervisorOne{Royyana Muslim Ijtihadie S.Kom, M.Kom., Ph.D}{198407082010122004}
\supervisorTwo{Bagus Jati Santoso, S.Kom., Ph.D}{198611252018031001}

\degree{Sarjana Komputer}{Arsitektur dan Jaringan Komputer}{S1}{Informatika}{Informatics}{Teknologi Informasi dan Komunikasi}{FTIK}{Information Technology and Communication}

\time{Juni}{2018}

\begin{document}
\renewcommand{\thelstlisting}{\arabic{chapter}.\arabic{lstlisting}}
	\frontmatter % Halaman dengan penomoran romawi kecil
	\maketitle
	\legalityPaper % Lembar Pengesahan
	
    \begin{abstrak}
		Saat ini, dengan didukung oleh konsep SaaS (Software as a Service), aplikasi web berkembang dengan pesat. Para penyedia layanan aplikasi web berlomba-lomba memberikan pelayanan yang terbaik, seperti menjaga QoS (Quality of Service) sesuai dengan perjanjian yang tertuang dalam SLA (Service Level Agreement). Hal tersebut dikarenakan permintaan akses ke suatu aplikasi web biasanya meningkat dengan seiring berjalannya waktu. Keramaian akses sesaat menjadi hal yang umum dalam aplikasi web saat ini. Saat hal tersebut terjadi, aplikasi web akan di akses lebih banyak dari kebiasaan. Jika aplikasi web tersebut tidak menyediakan kemampuan untuk menangani hal tersebut, bisa menyebabkan aplikasi web tidak dapat berjalan dengan semestinya yang sangat merugikan pengguna. \\
        \indent Elastic cloud merupakan salah satu bagian dari komputasi awan yang sedang populer, dimana banyak riset dan penelitian yang berfokus di bidang ini. Elastic cloud bisa digunakan untuk menyelesaikan permasalah di atas. Lalu sebuah perangkat lunak bernama Docker dapat dapat diterapkan untuk mendukung elastic cloud. \\
        \indent Dalam tugas akhir ini akan dibuat sebuah rancangan sistem yang memungkinkan aplikasi web berjalan di atas Docker. Sistem ini bisa beradaptasi sesuai dengan kebutuhan dari aplikasi yang sedang berjalan. Jika aplikasi membutuhkan sumber daya tambahan, sistem akan menyediakan sumber daya berupa suatu container baru secara otomatis dan juga akan mengurangi penggunaan sumber daya jika aplikasi sedang tidak membutuhkannya. Dari hasil uji coba, sistem dapat menangani sampai dengan 57.750 request dengan error request yang terjadi sebesar 7.83\%. \\

	\noindent \textbf{Kata-Kunci}: aplikasi web, autoscale, docker, elastic cloud
\end{abstrak}
\newpage
\begin{abstract}
		Nowdays, with the concept of SaaS (Software as a Service), web applications have developed a lot. Web service providers are competing to provide the best service, such as QoS (Quality of Service) requirements specified in the SLA (Service Level Agreement). The load of web applications usually very drastically along with time. Flash crowds are also very common in today's web applications world. When flash crowds happens, the web application will be accessed more than usual. If the web applications does not provide the ability to do so, it can make the web application not work properly which is very disadvantegeous to the users. \\
		\indent Elastic cloud is one of the most popular part of cloud computing, with much researchs in this subject. Elastic clouds can be used to solve the above problems. Then a Docker can be applied to support the elastic cloud. \\
		\indent In this final task will be made an application system that allows web applications running on top of Docker. This system can adjust according to the needs of the running applications. If the application requires additional resources, the system will automatically supply the resources of a new container and will also reduce resource usage if the application is not needing it. From the test results, the system can handle up to 57,750 requests and error ratio of 7.83\%. \\

	\noindent \textbf{Keywords}: autoscale, docker, elastic cloud, web application
\end{abstract}
	\chapter{Kata Pengantar}
		\begin{figure}[h]
			\centering
			\includegraphics[width=0.5\linewidth]{img/bismillah.png}
		\end{figure}

		Alhamdulillahirabbil’alamin, segala puji bagi Allah SWT, yang telah melimpahkan rahmat dan hidayah-Nya sehingga penulis dapat menyelesaikan Tugas Akhir yang berjudul \textbf{IMPLEMENTASI PENGENDALI ELASTISITAS SUMBER DAYA BERBASIS DOCKER UNTUK APLIKASI WEB}. Pengerjaan Tugas Akhir ini merupakan suatu kesempatan yang sangat baik bagi penulis. Dengan pengerjaan Tugas Akhir ini, penulis bisa belajar lebih banyak untuk memperdalam dan meningkatkan apa yang telah didapatkan penulis selama menempuh perkuliahan di Teknik Informatika ITS. Dengan Tugas Akhir ini penulis juga dapat menghasilkan suatu implementasi dari apa yang telah penulis pelajari.
		Selesainya Tugas Akhir ini tidak lepas dari bantuan dan dukungan beberapa pihak. Sehingga pada kesempatan ini penulis mengucapkan syukur dan terima kasih kepada:
		\begin{enumerate}
			\item Allah SWT atas anugerahnya yang tidak terkira kepada penulis dan Nabi Muhammad SAW.
			\item Ibu Henning Titi Ciptaningtyas, S.Kom., M.Kom selaku pembimbing I yang telah membantu, membimbing, dan memotivasi penulis mulai dari pengerjaan proposal hingga terselesaikannya Tugas Akhir ini.
			\item Bapak Bagus Jati Santoso, S.Kom., Ph.D selaku pembimbing II yang juga telah membantu, membimbing, dan memotivasi penulis mulai dari pengerjaan proposal hingga terselesaikannya Tugas Akhir ini.
			\item Darlis Herumurti, S.Kom., M.Kom., selaku Kepala Jurusan Teknik Informatika ITS pada masa pengerjaan Tugas Akhir, Bapak Radityo Anggoro, S.Kom., M.Sc., selaku koordinator TA, dan segenap dosen Teknik Informatika yang telah memberikan ilmu dan pengalamannya.
			\item Serta semua pihak yang telah turut membantu penulis dalam menyelesaikan Tugas Akhir ini.
		\end{enumerate}

		Penulis menyadari bahwa Tugas Akhir ini masih memiliki banyak kekurangan. Sehingga dengan kerendahan hati, penulis mengharapkan kritik dan saran dari pembaca untuk perbaikan ke depannya.

		\hfill Surabaya, Juni 2017 \\ \\
		\hfill Muhammad Fahrul Razi

	\cleardoublepage % Mengisi penanda halaman genap yang kosong

	\tableofcontents % Daftar Isi
	\listoftables % Daftar Tabel
	\listoffigures % Daftar Gambar
	\lstlistoflistings % Daftar Kode Sumber

	\mainmatter
	\chapter{PENDAHULUAN}
	Pada bab ini akan dipaparkan mengenai garis besar Tugas Akhir yang meliputi latar belakang, tujuan, rumusan dan batasan permasalahan, metodologi pembuatan Tugas Akhir, dan sistematika penulisan.
        
	\section{Latar Belakang}
		Saat ini, dengan didukung oleh konsep SaaS (\textit{Software as a Service}), aplikasi web berkembang dengan pesat. Banyak perusahaan, seperti Google, Amazon, dan Microsoft yang berhasil mencapai kesuksesan dari aplikasi web. Para penyedia layanan aplikasi web juga berlomba-lomba memberikan pelayanan yang terbaik, seperti menjaga QoS (\textit{Quality of Service}) sesuai dengan perjanjian yang tertuang dalam SLA (\textit{Service Level Agreement}) \cite{kan_docloud:_2016}. Hal tersebut dikarenakan permintaan akses ke suatu aplikasi web biasanya meningkat dengan seiring berjalannya waktu. Keramaian akses sesaat menjadi hal yang umum dalam aplikasi web saat ini. Saat hal tersebut terjadi, aplikasi web akan diakses lebih banyak dari keadaan biasanya. Jika aplikasi web tersebut tidak menyediakan kemampuan untuk menangani hal tersebut, bisa menyebabkan aplikasi web tidak dapat berjalan dengan semestinya yang sangat merugikan pengguna. Biasanya, pengembang akan melakukan pengaturan sumber daya \textit{server} secara manual agar bisa menangani permasalahan di atas, tapi akan memakan banyak biaya dan waktu. Tapi jika tidak ditangani, akibatnya aplikasi web tidak bisa berjalan saat pengalami puncak permintaan dari pengguna. Saat ini banyak tersedia layanan komputasi awan, yaitu sebuah model komputasi yang mana pengguna akan membayar sesuai dengan sumber daya yang digunakan. Dengan bantuan dari komputasi awan, penggembang bisa melakukan \textit{scale up} dan \textit{scale down} sumber daya \textit{server} dari aplikasi web secara manual atau memanfaatkan API yang disediakan oleh \textit{platform} yang bisa diakses dalam rentang waktu jam bahkan menit. Perkembangan dari komputasi awan melahirkan teknologi yang dikenal dengan \textit{autonomos elastic cloud}, sebuah sistem yang secara dinamis akan menambahkan sumber daya sesuai dengan jumlah permintaan. Saat permintaan akses ke suatu aplikasi web meningkat, \textit{elastic cloud} secara otomatis akan menambahkan sumber daya untuk aplikasi dan juga secara otomatis akan mengurangi sumber daya dari aplikasi saat permintaan aksesnya menurun.\\
		\indent \textit{Elastic cloud} merupakan salah satu bagian dari komputasi awan yang sedang populer, dimana banyak riset dan penelitian yang berfokus di bidang ini. Saat ini, biasanya \textit{elastic cloud} berbasis pada \textit{virtual machines} (VMs). VM sendiri dianggap terlalu berat untuk menjalankan sebuah aplikasi web, karena biasanya yang dibutuhakan oleh suatu aplikasi web hanya \textit{web server} (Apache, Nginx), bahasa pemrograman yang digunakan, basis data, dan komponen lainnya, tidak keseluruhan sistem operasi yang terjadi jika menggunakan VM. Dalam hal ini, menggunakan VM untuk mengembangkan aplikasi web hanya akan membuang-buang sumber daya dan menurunkan performa dari aplikasi. Selain itu, penerapan \textit{elastic cloud} yang berjalan di atas VM tidak bisa meningkatkan sumber daya dengan cepat yang bisa merusak QoS.\\
		\indent Sebuah perangkat lunak bernama \textit{docker} dapat menyelesaikan permasalahan dari VM. \textit{Docker} adalah sebuah perangkat lunak yang berfungsi sebagai wadah untuk membungkus dan memasukkan sebuah perangkat lunak ke dalam sebuah lingkungan beserta semua hal yang dibutuhkan oleh perangkat lunak tersebut. Selain membungkus aplikasi, \textit{docker} menjadikan aplikasi yang berjalan di atasnya menjadi terisolasi sehingga menghilangkan kemungkinan terjadinya kebocoran suatu proses aplikasi yang bisa menyebabkan kerusakan pada \textit{host}. \textit{Docker container} berjalan di atas \textit{host} dan menggunakan \textit{kernel} yang sama dengan \textit{host} yang mana memungkinkan \textit{container} dapat dibangun dengan cepat dan membuat penggunaan sumber daya menjadi lebih efisien.\\
		\indent Dalam tugas akhir ini akan dibuat sebuah rancangan sistem yang memungkinkan untuk menjalankan aplikasi web berbasis \textit{docker}. Sistem ini bisa beradaptasi sesuai dengan kebutuhan dari aplikasi yang sedang berjalan. Jika aplikasi membutuhkan sumber daya tambahan, sistem akan menyediakan sumber daya berupa suatu \textit{container} baru secara otomatis dan juga akan mengurangi penggunaan sumber daya jika aplikasi sedang tidak membutuhkannya. Proses skalabilitas ini termasuk skalabilitas secara horizontal, yaitu dengan menambah \textit{instance}, dalam kasus ini berupa \textit{docker container}, dari aplikasi web. Sistem ini juga menyediakan sebuah \textit{server docker repository} untuk menaruh aplikasi web dalam format \textit{docker}. Pengembang yang ingin memasang atau memperbarui aplikasinya di sistem ini akan melakukan \textit{push} aplikasi web dalam format \textit{docker} ke \textit{server repository} dan sistem secara otomatis akan membangun atau memperbarui aplikasi tersebut di \textit{server master host}.

	\section{Rumusan Masalah}
       	Rumusan masalah yang diangkat dalam tugas akhir ini adalah sebagai berikut :
		\begin{enumerate}
			\item Bagaimana cara membuat sistem yang dapat melakukan skalabilitas secara otomatis terhadap aplikasi web berbasis \textit{docker} dengan menggunakan \textit{Proactive Model} dan \textit{Reactive Model}?
			\item Bagaimana cara membuat sistem yang dapat mendistribusikan akses pengguna ke aplikasi web berbasis \textit{docker} secara efisien?
            \item Bagaimana cara membuat sistem yang dapat melakukan pembaruan untuk sebuah aplikasi web yang sudah berjalan tanpa terjadi \textit{downtime}?
		\end{enumerate}

	\section{Batasan Masalah}
		Dari permasalahan yang telah diuraikan di atas, terdapat beberapa batasan masalah pada tugas akhir ini, yaitu:
		\begin{enumerate}
			\item Semua \textit{container} dari aplikasi web akan berjalan hanya pada satu \textit{server master}.
            \item Perhitungan algoritma skala akan menggunakan \textit{Proactive Model} dan \textit{Reactive Model} untuk menentukan jumlah \textit{container} yang dibentuk atau dihapus.
            \item Aplikasi web yang diuji coba hanya akan melakukan komputasi tanpa terhubung dengan layanan luar, seperti koneksi ke suatu basis data dan layanan REST API.
		\end{enumerate}

	\section{Tujuan}
       	Tujuan dari pembuatan tugas akhir ini adalah membuat sistem yang dapat melakukan skalabilitas secara otomatis terhadap aplikasi web berbasis \textit{docker} dengan menggunakan \textit{Proactive Model} dan \textit{Reactive Model} untuk menentukan sumber daya yang diperlukan oleh aplikasi. Selain itu, sistem ini juga memiliki fitur \textit{hot-upgrade}, yaitu dapat melakukan pembaruan terhadap aplikasi yang sudah terpasang tanpa terjadi \textit{downtime}.
        
	\section{Manfaat}
		Tugas akhir ini diharapkan dapat memberikan kemudahan seorang pengembang aplikasi berbasis web dengan tidak perlu melakukan konfigurasi \textit{server} secara langsung untuk melakukan skalabilitas aplikasinya. Pengembang tidak perlu mengawasi aplikasinya saat terjadi perubahan permintaan yang tiba-tiba melonjak tinggi kemudian mengaturnya supaya bisa mengatasi permintaan tersebut. Sistem akan secara otomatis melakukan hal tersebut. Untuk menggunakan sistem ini, pengembang hanya perlu menyimpan aplikasinya di sebuah \textit{server docker repository} dan sistem akan mengelolanya lebih lanjut.

	\chapter{TINJAUAN PUSTAKA}
		\section{\textit{Publish/subscribe}}
			\textit{Publish/subscribe} muncul sebagai paradigma komunikasi yang populer untuk sistem terdistribusi dalam skala yang besar. dalam \textit{publish/subscribe} \textit{consumer} akan berlangganan ke suatu \textit{event} yang diinginkan. terlepas dari kegiatan \textit{consumer}, ada \textit{event producer} yang akan menerbitkan suatu \textit{event}. jika event yang diterbitkan oleh produser sesuai dengan \textit{event} yang dilanggani oleh \textit{consumer}, \textit{event} tersebut akan dikirim kepada \textit{consumer} secara \textit{asynchronus}. Interaksi ini difasilitasi oleh \textit{middleware publish/subscribe}. \textit{middleware publish/subscribe} dapat dipusatkan menjadi sebuah \textit{node} tunggal yang berperan sebagai \textit{broker} dari sebuah \textit{event} atau dipisahkan menjadi kumpulan beberapa \textit{node} \textit{broker} dari sebuah \textit{event}. 
			
			pada dasarnya, \textit{publish/subscribe} dibagi dari dua jenis yaitu: \textit{topic-based} dan \textit{content-based}. Pada \textit{topic-based publish-subscribe}, \textit{event} diterbitkan melalui sebuah topik dan \textit{consumer event} akan berlangganan topik tersebut untuk mendapatkan data dari suatu \textit{event}. berlangganan pada kasus \textit{topic-based} tidak didukung pemilahan data dari suatu \textit{event}. contohnya, \textit{consumer} akan menerima semua data dari suatu \textit{event} yang diterbikan pada suatu topik. pada \textit{content-based publish-subscribe}, berlangganan pada kasus ini didukung oleh fitur pemilahan yang diterapkan pada suatu \textit{event} yang diterbitkan. data yang dipilah oleh \textit{consumer} pada suatu \textit{event} yang berlangganan akan dikirimkan ke \textit{consumer}. \cite{noauthor_what_2016}
			
		\section{\textit{Websocket}}
          	Websocket adalah protokol berbasis TCP yang menyediakan \textit{channel} komunikasi \textit{full-duplex} antara \textit{server} dan \textit{client} melalui koneksi TCP tunggal. dibandingkan dengan skema komunikasi \textit{web real-time} tradisional, protokol Websocket menghemat banyak sumber daya \textit{bandwidth} pada jaringan, sumber daya server, dan performa \textit{real-time} yang sangat jauh lebih baik dibanding websocket tradisional. Websocket adalah protocol berbasis TCP yang independen. Websocket hanya berhubungan dengan HTTP yang memiliki \textit{handshake} yang diterjemahkan oleh HTTP \textit{server} sebagai pengembangan dari sebuah \textit{request}. Websocket terdiri dari dua bagian yaitu: \textit{handshake} dan \textit{data transfer}.
           		
           	Untuk membuat koneksi Websocket, \textit{client} harus mengirimkan \textit{request} HTTP kepada \textit{server}. Setelah itu protokol akan diupgrade menjadi protokol Websocket, lalu server akan mengenali tipe request berdasarkan header pada HTTP. Protokol akan diupgrade menjadi Websocket apalbila diminta oleh Websocket, dan kedua kubu (\textit{client} dan \textit{server}) akan memulai komunikasi \textit{full-duplex}, yang berarti \textit{client} dan \textit{server} dapat bertukar data kapanpun sampai salah satu dari \textit{client} atau \textit{server} menutup koneksi tersebut. Model komunikasi Websocket dapat dilihat pada gambar \ref{websocketmodel}.
           		
           	Websocket memiliki kemampuan yang lebih baik dalam berkomunikasi dibandingkan dengan skema komunikasi tradisional, dimana komunikasi terjadi secara \textit{realtime}. sekali koneksi sudah berhasil dibuat, \textit{server} dan \textit{client} melakukan aliran data dua arah, dimana aktivitas tersebut meningkatkan mempuan \textit{server} untuk mengirim data. Bandingkan dengan protokol HTTP, dimana informasi yang dikirimkan lebih ringkas dan mengurangi transmisi dari data yang redundan. Dengan skala user yang besar dan kebutuhan komunikasi \textit{realtime} yang tinggi, menurunkan beban pada jaringan akan menjadi keuntungan dibanding komunikasi \textit{realtime} secara tradisional.
           	\begin{figure}[H]
           		\centering
           		\includegraphics[height=10cm]{Images/C-2/websocket.png}
           		\caption{Model Komunikasi \textit{Websocket}}
           		\label{websocketmodel}
           	\end{figure}
           	\cite{boettiger_introduction_2015}.
		\section{SNMP}
			\textit{Simple Network Management Protocol} (SNMP) adalah aplikasi pada lapisan (\textit{layer}) protokol yang digunakan untuk mengatur data pada jaringan. hampir semua \textit{vendor} jaringan mendukung protokol SNMP. beberapa vendor peralatan telekomunikasi juga mulai didukung oleh protokol SNMP untuk mencapai pengaturan (manajemen) yang terintegrasi. Banyak dari aktivitas manajemen jaringan pada jaringan \textit{enterprise} yang menggunakan SNMP dalam persentasi yang sangat besar.
			
			SNMP yang berdasarkan paradigma \textit{server} - \textit{client} termasuk kedalam manajemen stasiun, agen dan \textit{Management Information Bases} (MIB). tujuan dari manajemen stasiun adalah untuk mengirimkan \textit{request} kepada \textit{agent} dan mengendalikan mereka, manajemen stasiun juga menyediakan antarmuka antara manajer jaringan manusia dan sistem manajemen jaringan. setiap perangkat jaringan memungkinkan untuk mempunyai agen yang dapat mengendalikan basis data dan ketika stasiun manajemen mulai melakukan polling, agen-agen tersebut akan mengirimkan laporan kepada stasiun manajemen.
			
			Dalam pendekatan pemantauan dengan menggunakan SNMP, setiap agen akan mengirimkan stasiun manajemen sebuah informasi melalui \textit{polling} laporan kejadian. \textit{Polling} adalah aktivitas untuk melakukan interaksi antara agen dan stasiun manajemen menggunakan metode \textit{request} dan \textit{response}. Namun, stasiun manajemen hanya mendengarkan kepada informasi masuk pada pendekatan pelaporan kejadian. Agen akan mengirim informasi kepada stasiun manajemen setiap informasi tersebut dibutuhkan berdasarkan sebuah keputusan.
			
			Pendekatan pemantauan secara \textit{realtime} akan didefinisikan sebagai persetujuan antara agen dan stasiun manajemen, dimana pada persetujuan semacam ini, agen harus mengirimkan informasi kepada manajemennya secara berkala tanpa permintaan dari stasiun.
			
			Dalam sebuah kelompok, status dan sifat dari sistem akan dipertimbangkan sebagai pekerjaan memantau dari informasi MIB. tipe data yang akan digunakan untuk memantau lebih penting dibandingkan dengan perancangan jaringan. berikut ini adalah ingormasi yang harus digunakan dalah pemantauan:
			\begin{itemize}
				\item Static: Struktur dan elemen didalam konfigurasi dikategorikan seperti id dari \textit{port} pada sebuah \textit{router} atau \textit{host}. Informasi ini akan jarang berubah.
				\item Dynamic: Informasi kejadian pada jaringan seperti paket dan elemen jaringan
				\item Statistical: Informasi harus berasal dari informasi dinamis seperti rata-rata dari paket yang dikirimkan pada tiap unit.
			\end{itemize}
		\subsection{OID}
			\textit{Object Identifier} adalah sesuatu untuk mengidentifikasi sebuah objek. Objek ini dapat berupa daerah atau \textit{disk drive} tunggal. Yang paling umum, didalam IEEE-RAC, adalah OUI (Organizationally Unique Identifier), dan diturunkan secara terorganisasir, dan terdaftar diluar OUI. pengidentifikasi yang paling umum selanjutnya, termasuk pengidentifikasi alamat ethernet adalah pengidentifikasi \textit{Extended Unique Identifiers} (EUI) atau the \textit{World Wide Name} (WWN). uniknya, untuk sistem yang sesuai, merupakan properti berharga dalam dua kasus ini. keunikan ini diasumsikan oleh struktur dari nomor unik yang dimulai dengan OUI. IEEE-RAC menetapkan OUI sebagai \textit{Object Identifier} untuk sebuah organisasi. \textit{Object Identifier} ini merupakan lapisan didalam konteks yang lebih luas dari pengidentifikasi yang diturunkan secara unik dari titik awal dari sebuah OID, \textit{International
			Telecommunication Union Telecommunication Standardization Sector} (ITU-T) dan dideskripsikan didalam standar ASN.1. jalur tersebut dilacak menuju ITU-T disebut sebagai "arc" dari sebuah OID. Arc ini berkembang menjadi OUI dan RAC lain menetapkan perancang dan melalui penempatan yang dibuat oleh organisasi hingga titik akhir dari sebuah \textit{Object Identifier}.
			\begin{figure}[H]
				\centering
				\includegraphics[width=10cm]{Images/C-2/OID.png}
				\caption{Contoh \textit{Object Identifier} (OID)}
				\label{oidexample}
			\end{figure}
		\section{Nagios}
			Nagios adalah perangkat lunak \textit{opensource} yang aktif dikembangkan, memiliki banyak user juga komunitas yang luas, memiliki banyak \textit{plugin} tambahan yang dikembangkan oleh user maupun yang terdapat langsung pada awal pengaturan, dan banyak buku tentang Nagios yang diterbitkan. Nagios juga merupakan sistem pemantauan yang paling yang paling populer yang cocok dengan hampir semua distribusi linux. Dukungan komersial tersedia dari perusahaan yang didirikan oleh penciptanya dan pengembang utama sebagai penyedia solusi resmi. Peralatan pemantauan yang berbasis nagios juga tersedia, seperti sensor yang dirancang untuk beroprasi bersama nagios. Karena fleksibilitas dari rancangan perangkat lunak yang menggunakan arsitektur \textit{plug-in}, layanan pengecekan untuk aplikasi yang pustakanya sudah ditentukan dapat di gunakan. Didalam nagios terdapat beberapa \textit{plugin} lain, seperti \textit{script} tambahan yang dapat dikostumisasi dan dapat digunakan pada nagios. Nagios adalah program yang ringan dan menyediakan alat pemantauan yang sempurna yang dapat membantu untuk memantau seluruh protokol yang aktif dan perakngkat jaringan yang terhubung dengan topologi. Nagios juga mampu untuk menyediakan grafik yang komperhensif dan bersifat \textit{realtime} dan analisis tren.
			
		\section{REST API}
			Istilah \textit{representational state transfer} (REST) diperkenalkan oleh Roy Fielding. Gaya arsitektur REST adalah arsitektur \textit{client-server} di mana \textit{client} mengirim \textit{request} ke \textit{server}, kemudian server memproses request dan mengembalikan \textit{response}. \textit{Request} dan \textit{response} ini membangun sekitar transfer dari representasi sumber daya. Sumber daya adalah sesuatu yang diidentifikasi oleh URI.
			
			REST lebih sederhana daripada SOAP. Bahasa REST didasarkan pada 
			penggunaan kata benda dan kata kerja. REST tidak perlu format pesan seperti \textit{envelope} dan \textit{header} yang diperlukan di SOAP. Jadi parsing XML juga tidak membutuhkan \textit{bandwidth} yang banyak. Prinsip desain REST adalah sebagai berikut: \textit{addressability, statelessness} dan \textit{uniform interface}.
			
			\textit{Addressability}-REST memodelkan dataset untuk beroperasi sebagai sumber daya di mana sumber daya ditandai dengan URI. Sebuah antarmuka yang seragam dan standar digunakan untuk mengakses sumber daya yang tersedia yaitu menggunakan metode HTTP yang tetap. Setiap transaksi bersifat independen dan tidak terkait dengan transaksi sebelumnya, karena semua
			data yang diperlukan untuk memproses permintaan terkandung dalam permintaan itu, data sesi \textit{client} tidak disimpan di sisi \textit{server}. Oleh karena itu tanggapan \textit{server} juga independen. 
			
			Prinsip membuat aplikasi REST sederhana dan ringan. Aplikasi web yang mengikuti arsitektur REST dapat disebut sebagai layanan web RESTful. Penggunaan layanan web menggunakan metode http GET, PUT, POST dan DELETE untuk mengambil, membuat, memperbaharui, dan menghapus sumber daya.
			
			Arsitektur REST dan contoh penggunaannya dapat dilihat pada gambar \ref{rest} dan \ref{rest2}.
			\begin{figure}[H]
				\centering
				\includegraphics[width=9cm]{Images/C-2/rest-api.jpg}
				\caption{Arsitektur REST}
				\label{rest}
			\end{figure}
			\begin{figure}[H]
				\centering
				\includegraphics[width=9cm]{Images/C-2/rest.png}
				\caption{Contoh Penggunaan REST}
				\label{rest2}
			\end{figure}
			
	\chapter{DESAIN DAN PERANCANGAN}
    Pada bab ini dibahas mengenai analisis dan perancangan sistem.
	
    \section{Kasus Penggunaan}
    	Terdapat dua aktor dalam sistem ini, yaitu Pengembang (Administrator) dan \textit{User} (Pengguna) dari aplikasi web yang dikelola oleh sistem. Diagram kasus penggunaan digambarkan pada Gambar \ref{usecase}.
        \begin{figure}[H]
			\centering
			\includegraphics[width=8cm,height=10cm]{Images/C-3/usecase.png}
			\caption{Diagram Kasus Penggunaan}
			\label{usecase}
		\end{figure}
        \indent Diagram kasus penggunaan pada Gambar \ref{usecase} dideskripsikan masing-masing pada Tabel \ref {tabelKodeKasusPenggunaan}.
        
        \begin{longtable}{|p{0.25\textwidth}|p{0.24\textwidth}|p{0.35\textwidth}|} % L = Rata kiri untuk setiap kolom, | = garis batas vertikal.
		    	
		    	% Kepala tabel, berulang di setiap halaman
		    	\caption{Daftar Kode Kasus Penggunaan} \label{tabelKodeKasusPenggunaan} \\
		    	\hline
		    	\textbf{Kode Kasus Penggunaan} & \textbf{Nama Kasus Penggunaan} & \textbf{Keterangan} \\ \hline
		    	\endhead
		    	\endfoot
		    	\endlastfoot
		    	UC-0001 & Manajemen Akun Pengguna. & Pengembang (Admin) dapat membuat, melihat, mengubah dan menghapus data akun pengguna. \\ \hline
		    	UC-0002 & Manajemen Perangkat Jaringan.  & Pengembang (Admin) dapat membuat, melihat, mengubah dan menghapus data perangkat jaringan.\\ \hline
		    	UC-0003 & Memantau Perangkat Jaringan. & Pengembang (Admin) dan Pengguna (User) dapat memantau seluruh perangkat jaringan yang sudah ia langgani. \\ \hline
		    	UC-0004 & Berlangganan Informasi Perangkat. & Pengembang (Admin) dan Pengguna (User) dapat berlangganan informasi perangkat jaringan yang diinginkan. \\ \hline
		    	UC-0005 & Berhenti Berlangganan Informasi Perangkat. & Pengembang (Admin) dan Pengguna (User) dapat berhenti berlangganan informasi perangkat jaringan yang diinginkan. \\ \hline	
		    \end{longtable}

	\section{Arsitektur Sistem}
		Pada sub-bab ini, dibahas mengenai tahap analisis dan kebutuhan bisnis dan desain dari sistem yang akan dibangun.

		\subsection{Desain Umum Sistem}
			Sistem yang akan dibuat yaitu sistem yang dapat melakukan pemantauan pada perangkat jaringan yang berbasis \textit{web} dengan metode \textit{pusblish/subscribe}, dimana pengguna (\textit{user}) harus berlangganan kepada suatu informasi untuk mendapatkan informasi yang diinginkan.
			
			Sistem ini melibatkan 3 (Tiga) server yang berfungsi sebagai web server dan 1 (satu) server yang berfungsi sebagai database server. Server aplikasi dan websocket server berada pada satu server, sehingga pada implementasinya webserver aplikasi dan websocket dijalankan pada port yang berbeda. Pada sistem ini \textit{client} yaitu pengguna (\textit{user}) dan pengelola (\textit{admin}) akan mengakses aplikasi menggunakan web browser. yang nantinya jika mengakses fitur selain memantau perangkat jaringan, aplikasi akan mengirimkan \textit{request} HTTP kepada REST API, dimana REST API tersebut melakukan transaksi data kepada database server. setelah itu REST API akan mengirimkan \textit{response} kepada aplikasi.
			
			jika client mengakses fitur memantau jaringan, maka aplikasi akan terhubung dengan websocket yang tugasnya mengakses data yang berada pada pub/sub server, dimana pubsub server menyimpan data yang diterbitkan oleh nagios, data tersebut adalah hasil response SNMP nagios kepada tiap perangkat jaringan terkait.
			Penjelasan secara umum arsitektur sistem akan diuraikan pada Gambar \ref{DesainUmumSistem}.
            \begin{figure}[H]
				\centering
				\includegraphics[width=9cm,height=8cm]{Images/C-3/main.png}
				\caption{Desain Umum Sistem}
				\label{DesainUmumSistem}
			\end{figure}

		\subsection{Desain REST API}
            	REST API bertujuan untuk menjadikan sistem yang memiliki performa yang baik, cepat dan mudah untuk di kembangkan (scale) terutama dalam pertukaran dan komunikasi data. REST API diakses menggunakan protokol HTTP. Penamaan dan struktur URL yang konsisten akan menghasilkan API yang baik dan mudah untuk dimengerti developer. URL API biasa disebut endpoint dalam pemanggilannya.
            	
            	Pada sistem ini terdapat beberapa endpoint, beberapa endpoint dibagi menjadi beberapa endpoint sesuai dengan perintah yang diajalankannya. misal: create, read, delete, update dan lain-lain. berikut ini adalah daftar endpoint yang dibutuhkan pada sistem ini:
            	\begin{enumerate}
            		\item "/"
            		\item "/register"
            		\item "/login"
					\item "/logout"
					\item "/users"
					\begin{enumerate}
						\item "/users/<id:string>"
						\item "/users/edit"
						\item "/users/delete"
					\end{enumerate}
					\item "/devices"
					\begin{enumerate}
						\item "/devices/<id:string>"
						\item "/devices/create"
						\item "/devices/edit"
						\item "/devices/delete"
					\end{enumerate}
					\item "/oid"
					\begin{enumerate}
						\item "/oid/create"
						\item "/oid/edit"
						\item "/oid/delete"
					\end{enumerate}
					\item "/subscribe"
					\begin{enumerate}
						\item "/subscribe/devices"
						\item "/subscribe/oid"
					\end{enumerate}
					\item "/unsubscribe"
					\begin{enumerate}
						\item "/unsubscribe/devices"
						\item "/unsubscribe/oid"
					\end{enumerate}           		
            	\end{enumerate}
            	
            	Server aplikasi mengirimkan HTTP request kepada REST API yang nantinya REST API akan melakukan trasaksi data pada database sesuai dengan endpointnya masing-masing. setelah itu REST API akan mengirimkan HTTP response kepada server aplikasi.
            	
            	Secara umum, arsitektur dari REST API dapat dilihat pada Gambar \ref{desain:desainrestapi}\\
                \begin{figure}[H]
                    \centering
                    \includegraphics[width=9cm]{Images/C-3/desainrestapi.png}
                    \caption{Desain REST API}
                    \label{desain:desainrestapi}
				\end{figure}
            
		\subsection{Desain Publisher Server}
			Pada publisher server, dipasang aplikasi untuk memantau kinerja jaringan, yaitu Nagios. Pada nagios, perdapat plugin untuk memantau kinerja jaringan dengan protokol SNMP yaitu check-snmp. plugin ini membutuhkan beberapa parameter, diantaranya: alamat perangkat yang ingin dipantau dan oid dari apa yang ingin dipantau.
			
			Sebuah script dibuat untuk mengambil data dan mengirimkannya menuju pub/sub server. setiap perangkat yang dipantau dimasukkan ke sebuah thread baru agar dapat berjalan secara paralel. didalam thread tersebut, setiap perangkat terkait diperiksa kinerjanya dengan protokol SNMP dan hasilnya dikirimkan kepada pub/sub server melalui sebuah exchange yang telah diikat dengan sebuah message queue yang sebelumnya telah diinisiasi. Rancangan umum dari \textit{Publisher Server} seperti yang digambarkan pada Gambar \ref{desain:desainpublisher}.
			
			Exchange yang dibuat oleh script tersebut namanya dibuat berdasarkan uuid versi 4 dari tiap device yang diambil dari database dan nama queue dibuat berdasarkan uuid versi 4 yang dibuat baru.
			
			\begin{figure}[H]
				\centering
				\includegraphics[width=9cm]{Images/C-3/desainpublisher.png}
				\caption{Desain Publisher Server}
				\label{desain:desainpublisher}
			\end{figure}
                
		\subsection{Desain Pub/Sub Server}
			Publish/subscribe server atau bisa juga disebut pub/sub server. yaitu sebuah server untuk menampung seluruh pesan yang dikirimkan oleh publisher. didalamnya terpasang aplikasi message broker yaitu RabbitMQ. seluruh pesan yang dikirimkan oleh publisher dikirimkan ke pub/sub server melalui sebuah exchange yang diikat dengan sebuah queue setelah itu server akan menyimpan pesan queue tersebut hingga ada consumer yang meminta data tersebut untuk dikirimkan. dalam kasus ini yang bertindak sebagai consumer adalah websocket server.
			
			Di sisi websocket dan server aplikasi, websocket menginisiasi sebuah exchange yang namanya dibuat berdasarkan uuid versi 4 dari tiap perangkat yang ingin dipantau dari database server, dengan syarat exchange dengan nama tersebut belum dibuat atau terdaftar sebelumnya. jika exchange dengan nama tersebut sudah dibuat atau terdaftar sebelumnya pada server maka websocket server tidak perlu membuat exchange tersebut. begitu juga dengan queue-nya. queue dibuat dengan nama uuid yang telah dibuat acak oleh client dengan algoritma uuid versi 4. setelah itu websocket baru mengambil data pereangkat sesuai dengan data apa saja yang dilanggani oleh client.
        	Secara umum, arsitektur rancangan dari \textit{Pub/Sub Server} dapat dilihat pada Gambar \ref{desain:desainpubsub}.
        	\begin{figure}[H]
				\centering
				\includegraphics[width=7cm,height=7cm]{Images/C-3/desainpubsub.png}
				\caption{Desain Pub/Sub Server}
				\label{desain:desainpubsub}
			\end{figure}
            

        \subsection{Desain Database Server}
        	\begin{figure}[H]
        		\centering
        		\includegraphics[width=9cm]{Images/C-3/desaindb.png}
        		\caption{Desain Database}
        		\label{desain:desaindatabase}
        	\end{figure}
        	
        	Desain database pada sistem ini adalah seperti yang digambarkan pada gambar \ref{desain:desaindatabase}. Terdapat tiga tabel utama yang mewakili tiap entitas yang terlibat dalam sistem ini, yaitu: users, devices, dan oid. selain itu, terdapat dua table many-to-many untuk menyimpan data pengguna yang telah berlangganan kepada tiap perangkat dan pengguna yang berlangganan kepada tiap OID (untuk mengetahui informasi apa saja yang ada pada tiap perangkat. tiap OID memiliki informasi yang berbeda).
        	
		\subsection{Desain Antarmuka}
			Desain antarmuka adalah desain untuk halaman yang nantinya akan digunakan oleh client baik itu pengguna (user) ataupun pengelola (admin). Antarmuka yang nantinya dibuat berbasis web dan  menggunakan Bootstrap 3 dan HTML. terdapat beberapa perbedaan pada antarmuka yang digunakan oleh pengelola dan pengguna. Misal, pada antarmuka yang digunakan pengguna tidak ada tombol untuk menghapus data perangkat, sedangkan pada antarmuka yang digunakan oleh pengelola terdapat tombol untuk menghapus data perangkat yang telah terdaftar.
			
			Desain antarmuka untuk menampilkan daftar seluruh perangkat yang tersedia pada sistem dapat dilihat pada gambar \ref{desain:antarmuka1}. pada halaman ini pengguna dan pengelola dapat meliahat daftar perangkat yang tersedia dan beberapa infonya, seperti: nama perangkat, alamat, dan lokasi dari perangkat tersebut. pada halaman in ipengguna dan pengolola juga bisa langsung berlangganan atau berhenti berlangganan dengan menekan sebuah tombol yang ada pada halaman ini.
        	\begin{figure}[H]
        		\centering
        		\includegraphics[width=9cm]{Images/C-3/antarmuka1.png}
        		\caption{Desain Antarmuka Menampilkan Daftar Perangkat Yang Tersedia}
        		\label{desain:antarmuka1}
        	\end{figure}
        
        	Desain antarmuka untuk menampilkan rincian dari antarmuka terkait dapat dilihat pada gambar \ref{desain:antarmuka2}. pada halaman ini pengguna dan pengolola dapat melihat seluruh rincian data yang ada pada perangkat. mulai dari nama perangkat, tipe perangkat, alamat perangkat, lokasi perangkat dan info apa saja yang dapat dipantau memalui OID.
        	
        	Pada halaman ini pengguna dan pengelola juga dapat belangganan dengan cara menekan sebuah tombol. tidak hanya berlangganan perangkatnya saja, pengguna dan pengelola juga dapat memilih untuk berlangganan info apa saja yang ingin didapatkan dari perangkat tersebut.
	        \begin{figure}[H]
	        	\centering
	        	\includegraphics[width=9cm]{Images/C-3/antarmuka2.png}
	        	\caption{Desain Antarmuka Menampilkan Rincian dari Perangkat Terkait}
	        	\label{desain:antarmuka2}
	        \end{figure}
        
        	Desain antamuka pemantauan pernangkat dapat dilihat pada gambar \ref{desain:antarmuka3}. pada halaman ini, pengguna dan pengelola akan mendapatkan data dari seluruh perangkat yang sudah dilanggani. data yang ditampilkan pada halaman ini dipilih berdasarkan info yang dipilih pada antarmuka menampilkan rincian dari antarmuka terkait yang bisa dilihat pada gambar \ref{desain:antarmuka2}.
		    \begin{figure}[H]
		    	\centering
		    	\includegraphics[width=9cm]{Images/C-3/antarmuka3.png}
		    	\caption{Desain Antarmuka Pemantauan Perangkat}
		    	\label{desain:antarmuka3}
		    \end{figure}

	\chapter{IMPLEMENTASI}
	Bab ini membahas implementasi sistem Pengendali Elastisitas secara rinci. Pembahasan dilakukan secara rinci untuk setiap komponen yang ada, yaitu: \textit{docker registry}, \textit{master host}, \textit{controller}, \textit{load balancer}, dan dasbor.
    
    \section{Lingkungan Implementasi}
    	Lingkungan implementasi dan pengembangan dilakukan menggunakan virtualisasi Proxmox dengan spesifikasi Host komputer adalah Intel(R) Core(TM) i3-2120 CPU @ 3.30GHz dengan memori 8 GB di Laboratorium Arsitektur dan Jaringan Komputer, Teknik Informatika ITS. Perangkat lunak yang digunakan dalam pengembangan adalah sebagai berikut:
        \begin{itemize}
        \item Sistem Operasi Linux Ubuntu Server 16.04 LTS
        \item RabbitMQ 3.7.5-1
        \item MySQL Ver 15.1 Distrib 10.0.34-MariaDB
        \item Python 2.7
        \item Flask 0.12.2
        \item Node.js v6.11.4
        \item Nagios 4.3.4
        \item Express.js 4.16.3
        \end{itemize}
        
	\section{Implementasi REST API}
    	REST API digunakan untuk memudahkan aplikasi agar ringan dan mudah untuk dikembangkan. pada tugas akhir ini, REST API memiliki fungsi utama untuk menyimpan data user yang berlangganan pada perangkat jaringan atau berlangganan OID (Informasi didalam perangkat jaringan). REST API dibangun dengan framework Python yaitu Flask dan dilengkapi ORM (Object-relational mapping) Database yaitu Peewee.
        \subsection{Pemasangan Python Flask dan Peewee}
        	Pemasangan \texttt{Python Flask} dapat dilakukan dengan mudah, cukup dengan memasangnya dengan manajer paket yang dimiliki oleh \texttt{Python} yaitu \texttt{Pip}. Setelah \texttt{Flask} berhasil terpasang, selanjutnya adalah tahap pemasangan ORM \texttt{Peewee}. \texttt{Peewee} adalah Object-relational Mapping dimana fungsi utamanya adalah memudahkan pengembang agar dapat menyambungkan aplikasi dengan database dan melakukan query dengan mudah. pemasangan ORM Peewee dapat dilakukan dengan cara mengambil berkas instalasinya pada git \url{https://github.com/coleifer/peewee.git} dan pasang Peewee sesuai dengan instruksi yang tertera pada situs git tersebut.
        	
        \subsection{Implementasi Endpoint pada REST API }
        	REST API diakses menggunakan protokol HTTP. Penamaan dan struktur URL yang konsisten akan menghasilkan API yang baik dan mudah untuk dimengerti developer. URL API biasa disebut endpoint dalam pemanggilannya.
        	
        	Pada sistem ini terdapat beberapa endpoint, beberapa endpoint dibagi menjadi beberapa endpoint sesuai dengan perintah yang diajalankannya. misal: create, read, delete, update dan lain-lain.
        	
        	Berikut ini adalah endpoint yang dibuat dalam sistem ini:
        	
        	\begin{longtable}{|p{0.05\textwidth}|p{0.40\textwidth}|p{0.13\textwidth}|p{0.25\textwidth}|} % L = Rata kiri untuk setiap kolom, | = garis batas vertikal.
        		
        		% Kepala tabel, berulang di setiap halaman
        		\caption{Daftar Endpoint pada REST API} \label{tabelEndpointRESTAPI} \\
        		\hline
        		\textbf{No} & \textbf{Endpoint (Route)} & \textbf{Metode} & \textbf{Aksi} \\ \hline
        		\endhead
        		\endfoot
        		\endlastfoot
        		1 & /register & POST & Membuat data baru pada tabel user di database \\ \hline
        		2 & /login & POST & Mengambil data pada tabel user dan mencocokkannya dengan JSON yang dikirimkan lewat body. setelah data username dan password cocok, lalu dibuatkan sebuah token JWT. \\ \hline
        		3 & /logout & POST & Memasukkan token JWT yang terdaftar pada server kedalam daftar hitam agar token tidak dapat digunakan lagi. \\ \hline
        		4 & /users & GET & Menampilkan seluruh data user yang terdaftar pada sistem \\ \hline
        		5 & /users/\textless{}string:username\textgreater{} & GET & Menampilkan data user berdasarkan username yang tertulis pada URL \\ \hline
        		6 & /devices/create & POST & Membuat data baru pada tabel devices di database \\ \hline
        		7 & /devices/edit/\textless{}string:id\textgreater{} & PUT & Mengubah data pada tabel devices di database yang ID nya sama dengan ID yang ada pada URL. \\ \hline
        		8 & /devices/delete & DELETE & Menghapus data pada tabel devices di database yang ID nya tertulis pada body yang bertipe JSON. \\ \hline
        		9 & /devices & GET & Menampilkan seluruh data perangkat yang terdaftar pada sistem \\ \hline
        		10 & /devices/\textless{}string:id\textgreater{} & GET & Menampilkan data user berdasarkan username yang tertulis pada URL \\ \hline
        		11 & /oid/create & POST & Membuat data baru pada tabel oid \\ \hline
        		12 & /oid/edit & POST & Mengubah data pada tabel oid di database yang ID nya tertulis pada body yang bertipe JSON. \\ \hline
        		13 & /oid/delete & POST & Menghapus data pada tabel oid di database yang ID nya tertulis pada body yang bertipe JSON. \\ \hline
        		14 & /subscribe/devices & POST & Membuat data baru pada tabel subscribe \\ \hline
        		15 & /unsubscribe/devices & POST & Menghapus data pada tabel subscribe di database yang ID nya tertulis pada body yang bertipe JSON. \\ \hline
        		16 & /subscribe/oid & POST & Membuat data baru pada tabel subscribe\_oid \\ \hline
        		17 & /unsubscribe/oid & POST & Menghapus data pada tabel subscribe\_oid di database yang ID nya tertulis pada body yang bertipe JSON. \\ \hline	
        	\end{longtable}
        	
    
    \section{Implementasi Publisher Server}
    	Publisher server merupakan server yang berfungsi untuk mengambil data pada perangkat jaringan secara berkala dan mengirimkannya menuju pub/sub server. publisher server menggunakan plugin \texttt{check\_snmp} bawaan program Nagios, Sehingga untuk melakukan pengambilan data, kita perlu memasang nagios pada server.
    	
    	setelah data berhasil dikumpulkan, data yang diambil pada tiap perangkat dikirimkan menuju pub/pub server melalui thread yang berbeda. proses ini dinamakan \texttt{multithreading}.
    		\subsection{Pemasangan Nagios Sebagai Pemantau dan Pengumpul Data Perangkat}
    			Pemasangan \texttt{Nagios} dapat dilakukan dengan beberapa cara, namun cara yang dipakai pada kasus ini adalah memasang \texttt{Nagios} langsung dari sumbernya untuk mendapatkan fitur terbaru, pembaharuan keamanan, dan pembetulan bug.
    			
    			Berikut ini adalah sumber untuk mendapatkan nagios yang siap untuk dipasang: \url{https://assets.nagios.com/downloads/nagioscore/releases/nagios-4.3.4.tar.gz}
    			
    			\texttt{Nagios} perlu beberapa perintah khusus yang hanyak bisa dilakukan oleh user yang bernama "nagios" maka dari itu diperlukan user pada server yang bernama "nagios" dengan nama group "nagcmd". Selain user, \texttt{Nagios} juga perlu beberapa paket yang harus terpasang sebelum memasang nagios itu sendiri. Beberapa paket diantaranya adalah: \texttt{build-essential, libgd2-xpm-dev, openssl, libssl-dev, unzip}
    			
    			Setelah \texttt{Nagios} terpasang, direktori kerja dari Nagios dapat dilihat pada direktori \texttt{/usr/local/nagios}

			\subsection{Pengumpulan Data dan Pembuatan Script Pengiriman}
				Untuk mengumpulkan data perangkat jaringan, dibutuhkan plugin bawaan nagios yang bernama \texttt{check\_snmp}. plugin tersebut berada pada direktori \texttt{/usr/local/nagios/libexec} untuk menjalankan plugin tersebut dibutuhkan dua parameter, yaitu: alamat dari perangkat jaringan yang ingin dipantau dan OID dari data yang ingin didapatkan dari perangkat.
				perintah yang dijalankan untuk mendapatkan data pada perangkat jaringan lewat protokol SNMP adalah seeprti yang tertulis pada kode sumber 
				
\begin{lstlisting}[frame=single,breaklines,caption={Perintah Mengumpulkan Data Perangkat dengan SNMP},label=snmpcommand, captionpos=b]
$ /usr/local/nagios/check_snmp -H <alamat_perangkat> -o <oid_perangkat>
\end{lstlisting}
    			
    			Setelah data dapat dikumpulkan, sebuah script diperlukan untuk mengirim data tersebut menuju pub/sub server yang didukung boleh RabbitMQ sebagai Message Broker.
    			
    			Sebuah library bernama \texttt{pika} dibutuhkan untuk mengirim data tersebut ke pub/sub server. tiap perangkat yang dikumpulkan datanya dan dikirimkan ke pub/sub server, diproses didalam sebuah thread yang berbeda. oleh karena itu inisiasi database dibutuhkan pada awal script untuk menggetahui ada berapa perangkat yang terdaftar pada sistem.
    			
    			pertama-tama, masukkan library yang dibutuhkan untuk pembuatan script (termasuk pika), lalu dilanjutkan dengan potongan kode untuk menginisiasi database. Pseudocode untuk inisasi kelas database dapat dilihat pada kode sumber \ref{pseudo:dbclass}
    			
\begin{lstlisting}[frame=single,breaklines,caption={Pseudocode inisiasi Kelas Database},label=pseudo:dbclass, captionpos=b]
class BaseModel(Model):
class Meta:
database = database

class Users(BaseModel):
id = UUIDField(primary_key=True)
name = CharField()
username = CharField(unique=True)
password = CharField()
email = CharField()
role = CharField()

class Devices(BaseModel):
id = UUIDField(primary_key=True)
name = CharField()
type = CharField()
location = CharField()
address = CharField()

class Oid(BaseModel):
id = UUIDField(primary_key=True)
oid = CharField()
oidname = CharField()
devices_id = ForeignKeyField(Devices, on_delete='CASCADE')

class Subscribe(BaseModel):
users_id = ForeignKeyField(Users, on_delete='CASCADE')
devices_id = ForeignKeyField(Devices, on_delete='CASCADE')
\end{lstlisting}
    			
    			Setelah itu buat fungsi sebagai target menjalankan thread, nantinya tiap thread akan mengeksekusi kode yang ada didalam fungsi tersebut. didalam fungsi tersebut meliputi pegumpulan data dengan \texttt{check\_snmp}. Data perangkat jaringan yang dikumpulkan dengan \texttt{check\_snmp} dimasukkan kedalam sebuah \texttt{python dictionary} yang nantinya dictionary tersebut akan dikirimkan menuju pub/sub server. Pseudocode fungsi tersebut dapat dilihat pada kode sumber \ref{pseudo:threadtarget}
    			
\begin{lstlisting}[frame=single,breaklines,caption={Pseudocode Target \textit{Thread} Untuk Mengambil Data Perangkat},label=pseudo:threadtarget, captionpos=b]
rabbitMq(exchange, address):
	try:
		add getOidData() into array of dictionary
	except:
		add NULL into array of dictionary
	
	try:
		add getSnmpDeviceData() into JSON
	except:
		add Error Message into JSON		
\end{lstlisting}
    			
    			Untuk mengirimkan data menuju pub/sub server diperlukan library pika yang akan membuat koneksi dengan RabbitMQ yang berada di pub/sub server. potongan kode untuk mengirimkan data tersebut dapat dilihat pada kode sumber \ref{pseudo:pika}
    			
\begin{lstlisting}[frame=single,breaklines,caption={Pseudocode Pengiriman Data Dengan Pika},label=pseudo:pika, captionpos=b]
pika.openConnection()
if exchange does not exist:
	createExchage()
	if queue does not exist:
		createQueue()
		bindExchangetoQueue()
	else:
		bindExchangetoQueue()
else:
	pass
sendMessage()
\end{lstlisting}
    			
				Setelah seluruh fungsi selesai dibuat, langkah terakhir adalah membuat thread agar tiap thread nantinya akan mejalankan fungsi yang telah dibuat dan menjalankannya secara berkala. Pseudocode untuk membuat thread dapat dilihat pada kode sumber \ref{pseudo:runthread}
				
\begin{lstlisting}[frame=single,breaklines,caption={Pseudocode Menjalankan Thread},label=pseudo:runthread, captionpos=b]
Thread
while true:
	getDeviceId() as exchangename
	getDeviceAddress as deviceaddress
	thread(target=rabbitmq(), argument=(exchangename, deviceaddress))
	sleep(2)

\end{lstlisting}

    \section{Implementasi Pub/Sub Server}
    	Pada pub/sub server, dipasang aplikasi message broker \texttt{RabbitMQ}. pada kasus ini \texttt{RabbitMQ} menerima seluruh dapat yang dikirmkan oleh publisher. Setelah itu \texttt{RabbitMQ} menyimpannya dan menunggu hingga ada consumer yang meminta data pada \texttt{RabbitMQ}. kriteria data yang dikirimkan harus dispesifikkan sesuai dengan apa yang diminta oleh consumer.
    	
    	Pemasangan aplikasi \texttt{RabbitMQ} membutuhkan bahasa pemrograman \texttt{erlang}. untuk itu sebelum memasang \texttt{RabbitMQ}, harus terlebih dahulu memasang \texttt{erlang} pada sistem. Selain \texttt{erlang}, beberapa paket juga harus terpasang pada sistem, beberapa diantaranya adalah: \texttt{init-system-helpers, socat, adduser, logrotate}
    	
    	Setelah \texttt{RabbitMQ} server terpasang, selanjutnya dibutuhkan sebuah web admin untuk \texttt{RabbitMQ} agar mudah untuk melakukan manajemen data, user dan lain-lain pada web admin tersebut. \texttt{RabbitMQ} sudah menyediakan \textit{plugin} agar web admin dapat langsung digunakan. hanya dengan menjalankan perintah untuk mengaktifkan web admin dengan \texttt{rabbitmqctl}
    
    \section{Implementasi Consumer pada Application Server dan Websocket}
		Sebagai media penyimpn
    \section{Implementasi Database Server}
    	Sebagai media penyimpanan, sebuah database diperlukan untuk menyimpan data pengguna, perangkat, dan data berlangganan. Terdapat tiga tabel utama yang mewakili tiap entitas yang terlibat dalam sistem ini, yaitu: \texttt{users}, \texttt{devices}, dan \texttt{oid}. selain itu, terdapat dua table \texttt{many-to-many} untuk menyimpan data pengguna yang telah berlangganan kepada tiap perangkat dan pengguna yang berlangganan kepada tiap OID (untuk mengetahui informasi apa saja yang ada pada tiap perangkat. tiap OID memiliki informasi yang berbeda).
    	
    	Pada Tugas Akhir ini, sistem basis data yang digunakan adalah \texttt{Mysql Server} yang dimana \texttt{Mysql Server} termasuk kedalam RDBMS (\textit{Relational Database Management System}). Berikut adalah rincian dari tabel yang diimplementasikan. rincian tabel \texttt{users} dapat dilihat pada tabel \ref{tabeldbusers}
    	
    	\begin{longtable}{|p{0.05\textwidth}|p{0.20\textwidth}|p{0.22\textwidth}|p{0.35\textwidth}|} % L = Rata kiri untuk setiap kolom, | = garis batas vertikal.
    		
    		% Kepala tabel, berulang di setiap halaman
    		\caption{Rincian Tabel \texttt{users} pada Database} \label{tabeldbusers} \\
    		\hline
    		\textbf{No} & \textbf{Kolom} & \textbf{Tipe Data} & \textbf{Keterangan} \\ \hline
    		\endhead
    		\endfoot
    		\endlastfoot
    		1 & id & varchar(255) & Sebagai primary key pada tabel, nilai pada kolom ini berformat UUID versi 4 \\ \hline
    		2 & name & varchar(255) & Data yang berbentuk string. Digunakan untuk kelengkapan profil pengguna. \\ \hline
    		3 & username & varchar(255) & Data yang berbentuk string. Digunakan untuk keperluan autentikasi. \\ \hline
    		4 & password & varchar(255) & Data yang berbentuk string, implementasinya berupa hash. Digunakan untuk keperluan autentikasi. \\ \hline
    		5 & email & varchar(255) & Data yang berbentuk string. Digunakan untuk kelengkapan profil pengguna \\ \hline
    		6 & role & varchar(255) & Data yang berbentuk string. Digunakan untuk kelengkapan profil pengguna dan pembeda peran agar setiap user memiliki hak istimewa masing-masing. \\ \hline
    	\end{longtable}
    	
    	\begin{longtable}{|p{0.05\textwidth}|p{0.20\textwidth}|p{0.22\textwidth}|p{0.35\textwidth}|} % L = Rata kiri untuk setiap kolom, | = garis batas vertikal.
    		
    		% Kepala tabel, berulang di setiap halaman
    		\caption{Rincian Tabel \texttt{devices} pada Database} \label{tabeldbdevices} \\
    		\hline
    		\textbf{No} & \textbf{Kolom} & \textbf{Tipe Data} & \textbf{Keterangan} \\ \hline
    		\endhead
    		\endfoot
    		\endlastfoot
    		1 & id & varchar(255) & Sebagai primary key pada tabel, nilai pada kolom ini berformat UUID versi 4 \\ \hline
    		2 & name & varchar(255) & Data yang berbentuk string. Digunakan untuk kelengkapan profil dari perangkat jaringan. \\ \hline
    		3 & type & varchar(255) & Data yang berbentuk string. Digunakan untuk kelengkapan profil dari perangkat jaringan. \\ \hline
    		4 & location & varchar(255) & Data yang berbentuk string. Digunakan untuk kelengkapan profil dari perangkat jaringan. \\ \hline
    		5 & address & varchar(255) & Data yang berbentuk string, implementasinya barbentuk alamat IP dari tiap perangkat jaringan. Digunakan untuk mengkoleksi data pada publisher server. \\ \hline
    	\end{longtable}
    
    	\begin{longtable}{|p{0.05\textwidth}|p{0.20\textwidth}|p{0.22\textwidth}|p{0.35\textwidth}|} % L = Rata kiri untuk setiap kolom, | = garis batas vertikal.
    		
    		% Kepala tabel, berulang di setiap halaman
    		\caption{Rincian Tabel \texttt{OID} pada Database} \label{tabeldboid} \\
    		\hline
    		\textbf{No} & \textbf{Kolom} & \textbf{Tipe Data} & \textbf{Keterangan} \\ \hline
    		\endhead
    		\endfoot
    		\endlastfoot
    		1 & id & varchar(255) & Sebagai primary key pada tabel, nilai pada kolom ini berformat UUID versi 4 \\ \hline
    		2 & oid & varchar(255) & Data yang berbentuk string, implementasinya berbentuk OID (Object-Identifier). Digunakan mengkoleksi perangkat jaringan pada publisher server. \\ \hline
    		3 & oidname & varchar(255) & Data yang berbentuk string. Digunakan untuk kelengkapan profil dari perangkat jaringan. \\ \hline
    		4 & devices\_id & varchar(255) & Merupakan foreign key dari id pata tabel \texttt{devices}. Data ini berbentuk string, nilai pada kolom ini berformat UUID versi 4. \\ \hline
    	\end{longtable}
    
    	\begin{longtable}{|p{0.05\textwidth}|p{0.20\textwidth}|p{0.22\textwidth}|p{0.35\textwidth}|} % L = Rata kiri untuk setiap kolom, | = garis batas vertikal.
    		
    		% Kepala tabel, berulang di setiap halaman
    		\caption{Rincian Tabel \texttt{subscribe} pada Database} \label{tabeldbsubscribe} \\
    		\hline
    		\textbf{No} & \textbf{Kolom} & \textbf{Tipe Data} & \textbf{Keterangan} \\ \hline
    		\endhead
    		\endfoot
    		\endlastfoot
    		1 & users\_id & varchar(255) & Sebagai primary key pada tabel juga sebagai foreign key dari id pada tabel \texttt{users}, nilai pada kolom ini berformat UUID versi 4 \\ \hline
    		2 & devices\_id & varchar(255) & Sebagai primary key pada tabel juga sebagai foreign key dari id pada tabel \texttt{devices}, nilai pada kolom ini berformat UUID versi 4. \\ \hline
    	\end{longtable}
    
    	\begin{longtable}{|p{0.05\textwidth}|p{0.20\textwidth}|p{0.22\textwidth}|p{0.35\textwidth}|} % L = Rata kiri untuk setiap kolom, | = garis batas vertikal.
    		
    		% Kepala tabel, berulang di setiap halaman
    		\caption{Rincian Tabel \texttt{subscribeoid} pada Database} \label{tabeldbsubscribeoid} \\
    		\hline
    		\textbf{No} & \textbf{Kolom} & \textbf{Tipe Data} & \textbf{Keterangan} \\ \hline
    		\endhead
    		\endfoot
    		\endlastfoot
    		1 & users\_id & varchar(255) & Sebagai primary key pada tabel juga sebagai foreign key dari id pada tabel \texttt{users}, nilai pada kolom ini berformat UUID versi 4 \\ \hline
    		2 & oid\_id & varchar(255) & Sebagai primary key pada tabel juga sebagai foreign key dari id pada tabel \texttt{oid}, nilai pada kolom ini berformat UUID versi 4. \\ \hline
    	\end{longtable} 

    \section{Implementasi Antarmuka}
    	Dasbor diimplementasikan menggunankan perangkat kerja React bagian \textit{frontend} dan Flask untuk \textit{backend}nya. Dasbor digunakan untuk mempermudah pengembang dalam mengelola aplikasi. Dasbor memiliki menu-menu sebagai berikut:
        \begin{itemize}
        \item Daftar Aplikasi
        \item Infromasi Aplikasi
        \item Daftar \textit{Container}
        \item Matrik Aplikasi
        \end{itemize}
        \indent Masing-masing menu berikutnya akan dijelaskan secara rinci.
		\subsection{Daftar Aplikasi}
        	Daftar aplikasi, juga merupakan beranda dari dasbor, adalah menu yang digunakan untuk melihat daftar aplikasi atau \textit{image} yang ada pada \textit{server docker registry}. Pada halaman ini, bisa dilihat nama beserta versi terakhir dari aplikasi. Lalu juga terdapat status apakah aplikasi sedang berjalan atau tidak. Antar muka kelola daftar aplikasi ditunjukkan pada Gambar \ref{ddaftaraplikasi}.
			\begin{figure}[H]
				\centering
				\includegraphics[width=11.2cm,height=3.7cm]{Images/C-4/dasberanda.PNG}
				\caption{Dasbor Daftar Aplikasi}
				\label{ddaftaraplikasi}
			\end{figure}
            
         \subsection{Informasi Aplikasi}
         	\begin{figure}[H]
				\centering
				\includegraphics[width=11.2cm,height=9cm]{Images/C-4/dasinformasi.PNG}
				\caption{Dasbor Informasi Aplikasi}
				\label{dinformasiaplikasi}
			\end{figure}
         	Halaman ini menunjukkan infromasi lengkap dari sebuah aplikasi. Pada halaman ini bisa dilihat nama aplikasi, port yang digunakan, domain dari aplikasi, dan versi dari aplikasi. Pada halaman ini, pengguna bisa mengatur port dari aplikasi agar dapat berjalan dengan baik. Dari halaman ini juga aplikasi pertama kali akan dijalankan. Jadi pada halaman ini terdapat kontrol untuk menajalankan dan mematikan aplikasi. Antar muka informasi ditunjukkan pada Gambar \ref{dinformasiaplikasi}.
            
         \subsection{Daftar \textit{Container}}
         	Pada halaman ini, pengguna dapat melihat daftar \textit{container} yang sedang berjalan untuk sebuah aplikasi. Informasi yang diberikan berupa ID dan port dari container. Antar muka halaman daftar \textit{container} ditunjukkan pada Gambar \ref{ddaftarcontainer}.
            \begin{figure}[H]
				\centering
				\includegraphics[width=11.2cm,height=3cm]{Images/C-4/dasdafcont.PNG}
				\caption{Dasbor Daftar \textit{Container}}
				\label{ddaftarcontainer}
			\end{figure}
            
         \subsection{Metrik Aplikasi}
         	Halaman metrik aplikasi digunakan untuk memantau keadaan sekarang dari aplikasi. Metrik yang diberikan adalah jumlah \textit{request} pengguna ke aplikasi dan jumlah \textit{container} dari aplikasi. Data akan diperbarui sekitar lima detik sekali. Antar muka metrik aplikasi ditunjukkan pada Gambar \ref{dmatrikaplikasi}.
            \begin{figure}[H]
				\centering
				\includegraphics[width=11.2cm,height=7.3cm]{Images/C-4/dasmatrik.PNG}
				\caption{Dasbor Matrik Aplikasi}
				\label{dmatrikaplikasi}
			\end{figure}
	\chapter{PENGUJIAN DAN EVALUASI}

\section{Lingkungan Uji Coba}
	Lingkungan pengujian menggunakan komponen-komponen yang terdiri dari: satu \textit{server publisher}, satu \textit{server publish/subscribe}, satu \textit{server aplikasi}, satu \textit{server API}, satu \textit{server database} dan satu komputer penguji. Semua \textit{server} menggunakan vitual machine yang dipasang pada Virtual Box. Lalu, untuk komputer penguji menggunakan satu buah desktop sebagai \textit{docker} klien yang digunakan untuk menerima data yang dikirim open publisher. Pengujian dilakukan di Laboratoriom Arsitektur dan Jaringan Komputer Jurusan Teknik Informatika ITS. \\
    \indent Spesifikasi untuk setiap komponen yang digunakan ditunjukkan pada Tabel \ref{spesifikasikomponen}.
    \begin{longtable}{|p{0.05\textwidth}|p{0.18\textwidth}|p{0.33\textwidth}|p{0.33\textwidth}|}					\caption{Spesifikasi Komponen} \label{spesifikasikomponen} \\
        \hline
        \textbf{No} & \textbf{Komponen} & \textbf{Perangkat Keras} & \textbf{Perangkat Lunak} \\ \hline
        \endfirsthead
        \caption[]{Spesifikasi Komponen} \\
        \hline
        \textbf{No} & \textbf{Komponen} & \textbf{Perangkat Keras} & \textbf{Perangkat Lunak} \\ \hline
        \endhead
        \endfoot
        \endlastfoot

    	1 & Publisher & 2 core processor, 4GB RAM, 20GB SSD & Ubuntu 16.04 LTS, Python2.7, Nagios \\ \hline
        2 & Pub/Sub & 8 core processor, 16GB RAM, 20GB SSD & Ubuntu 16.04 LTS, RabbitMQ \\ \hline
        3 & Application & 2 core processor, 4GB RAM, 20GB SSD & Ubuntu 16.04 LTS, Node.JS, Python 2.7 \\ \hline
        4 & REST API & 1 core processor, 512MB RAM, 20GB SSD & Ubuntu 16.04 LTS, Docker 17.03.0-ce, Python 2.7 \\ \hline
        5 & Database & 1 core processor, 512MB RAM, 20GB SSD & Ubuntu 16.04 LTS, MySQL Server \\ \hline
        6 & Komputer penguji & Processor Core2Duo E7300, 2GB RAM & Windows 8, JMeter 3.2 \\ \hline
    \end{longtable}
    
    \indent Untuk akses ke masing-masing komponen, digunakan IP private yang disediakan untuk masing-masing komponen tersebut. Detailnya ditunjukkan pada Tabel \ref{ipdomainserver}.
    			\begin{longtable}{|p{0.05\textwidth}|p{0.33\textwidth}|p{0.44\textwidth}|}					\caption{IP dan Domain Server} \label{ipdomainserver} \\
					\hline
					\textbf{No} & \textbf{Server} & \textbf{IP dan Domain} \\ \hline
					\endfirsthead
					\caption[]{IP dan Domain Server} \\
					\hline
					\textbf{No} & \textbf{Server} & \textbf{IP dan Domain} \\ \hline
					\endhead
					\endfoot
					\endlastfoot
					
                    1 & Publisher & 128.199.160.188 \\ \hline
                    2 & Pub/Sub & 128.199.182.29 \\ \hline
                    3 & Application & 128.199.250.137 \\ \hline
                    4 & REST API & 139.59.97.244 \\ \hline
                    5 & Database & 139.59.97.244 \\ \hline
				\end{longtable}
    
\section{Skenario Uji Coba} \label{skenarioujicoba}
	Uji coba akan dilakukan untuk mengetahui keberhasilan sistem yang telah dibangun. Skenario pengujian dibedakan menjadi 2 bagian, yaitu:
    \begin{itemize}
    \item \textbf{Uji Fungsionalitas} \\
    	Pengujian ini didasarkan pada fungsionalitas yang disajikan sistem.
    \item \textbf{Uji Performa} \\
    	Pengujian ini untuk menguji kecepatan respon sistem terhadap sejumlah permintaan ke aplikasi secara bersamaan. Pengujian dilakukan dengan melakukan \textit{benchmark} pada sistem.
    \end{itemize}
    
    \subsection{Skenario Uji Coba Fungsionalitas}
    	Uji fungsionalitas dibagi menjadi 2, yaitu uji fungsionalitas antarmuka aplikasi dan uji fungsionalitas endpoint REST API.
        
        \subsubsection{Uji Fungsionalitas Endpoint Antarmuka Aplikasi} \label{ujimengelolaaplikasiberbasisdocker}
        	Pengujuian ini dilakukan untuk memeriksa apakah semua fungsi yang berada pada aplikasi dapat dijalankan dengan benar. Pengujian dilakukan dengan cara mengakses antarmuka yang berhubungan dengan tugas akhir ini lalu menjalankan fitur yang disediakan pada tiap-tiap antarmuka.
        	
        	Rancangan pengujian dan hasil yang diharapkan dapat dilihat pada tabel \ref{ujiaplikasi}.
        	
            \begin{longtable}{|p{0.05\textwidth}|p{0.20\textwidth}|p{0.30\textwidth}|p{0.27\textwidth}|}					\caption{Skenario Uji Fungsionalitas Antarmuka Aplikasi} \label{ujiaplikasi} \\
					\hline
					\textbf{No} & \textbf{Fitur} & \textbf{Uji Coba} & \textbf{Hasil Harapan} \\ \hline
					\endfirsthead
					\caption[]{Skenario Uji Fungsionalitas Antarmuka Aplikasi} \\
					\hline
					\textbf{No} & \textbf{Fitur} & \textbf{Uji Coba} & \textbf{Hasil Harapan} \\ \hline
					\endhead
					\endfoot
					\endlastfoot
					
                    1 & Autentikasi pengguna untuk masuk kedalam sistem. & Pengguna memasukkan username dan password masing-masing milik pengguna pada form yang telah disediakan. & Pengguna dapat masuk ke halaman utama aplikasi setelah menekan tombol "login"\\ \hline
                    2 & Menampilkan seluruh data perangkat. & Pengguna menekan \textit{menu} "DEVICE MANAGEMENT" pada \textit{sidebar}. & Sistem menampilkan seluruh data perangkat yang terdaftar pada sistem. data yang ditampilkan meliputi: nama perangkat, tipe perangkat, alamat perangkat dan lokasi perangkat. serta terdapat tombol informasi untuk melihat data masing-masing perangkat secara rinci dan tombol hapus untuk menghapus data perangkat yang terdaftar pada sistem. \\ \hline
                    3 & Menampilkan rincian data perangkat & Pengguna menekan tombol informasi yang tersedia pada tabel pada halaman menampilkan seluruh data perangkat. & Sistem menampilkan data perangkat terkait secara rinci. data yang ditampilkan meliputi: profil perangkat, pelanggan dari perangkat dan OID (Informasi yang disediakan pada perangkat terkait). \\ \hline
                    4 & Menghapus data perangkat. & Pengguna menekan tombol hapus yang tersedia pada tabel pada halaman menampilkan seluruh data perangkat. & Sistem menghapus data perangkat terkait secara permanen. \\ \hline
					5 & Menyunting data perangkat. & Pengguna menekan tombol ubah data pada halaman rincian data perangkat lalu mengubah data yang tersedia pada form yang berisi data sebelumnya. & Sistem mengubah data yang lama dengan data yang baru dimasukkan oleh pengguna. \\ \hline
                    6 & Berlangganan Data Perangkat. & Pengguna menekan tombol "Subscribe" yang tersedia pada halaman rincian data perangkat. & Sistem menandai bahwa perangkat atau OID (Informasi pada perangkat) yang terkait telat dilanggani. tombol akan berubah menjadi "Unsubscribe"\\ \hline
                    7 & Memantau kondisi perangkat yang telah dilanggani. & Pengguna menekan \textit{menu} "MONITOR" pada \textit{sidebar}. & Sistem menampilkan kondisi dari seluruh perangkat yang telah dilanggani informasinya oleh pengguna. \\ \hline
				\end{longtable}
            
        \subsubsection{Uji Fungsionalitas Endpoint REST API}
        	Pengujian ini dilakukan untuk memeriksa apakah seluruh fungsi dan \textit{endpoint} yang berhubungan dengan tugas akhir ini dan tersedia pada sistem dapat bekerja dengan semestinya. Pengujian dilakukan dengan cara mengakses \textit{endpoint} dengan metode tertentu disertai dengan \textit{header} autentikasi JWT . Rancangan pengujian dan hasil yang diharapkan ditunjukkan dengan Tabel \ref{ujirestapi}.
            
            \begin{longtable}{|p{0.05\textwidth}|p{0.20\textwidth}|p{0.30\textwidth}|p{0.27\textwidth}|}					\caption{Skenario Uji Fungsionalitas REST API} \label{ujirestapi} \\
            	\hline
            	\textbf{No} & \textbf{\textit{Endpoint}} & \textbf{Uji Coba} & \textbf{Hasil Harapan} \\ \hline
            	\endfirsthead
            	\caption[]{Skenario Uji Fungsionalitas REST API} \\
            	\hline
            	\textbf{No} & \textbf{\textit{Endpoint}} & \textbf{Uji Coba} & \textbf{Hasil Harapan} \\ \hline
            	\endhead
            	\endfoot
            	\endlastfoot
            	
            	1 & /login. & Mengakses endpoint dengan header autentikasi JWT dan metode POST, disertai dengan body bertipe JSON yang dilengkapi dengan beberapa parameter seperti: username dan password.\\ \hline
            	2 & /devices. & Mengakses endpoint dengan header autentikasi JWT dan metode GET. & Sistem menampilkan seluruh data perangkat yang terdaftar pada sistem. data yang ditampilkan meliputi: nama perangkat, tipe perangkat, alamat perangkat dan lokasi perangkat. serta terdapat tombol informasi untuk melihat data masing-masing perangkat secara rinci dan tombol hapus untuk menghapus data perangkat yang terdaftar pada sistem. \\ \hline
            	3 & /devices/ <string:id> & Mengakses endpoint dengan header autentikasi JWT, metode GET dan menyertakan ID perangkat pada endpoint. & Sistem menampilkan data perangkat terkait secara rinci. data yang ditampilkan meliputi: profil perangkat, pelanggan dari perangkat dan OID (Informasi yang disediakan pada perangkat terkait). \\ \hline
            	4 & /devices /create. & Mengakses endpoint dengan header autentikasi JWT dan metode POST, disertai dengan body bertipe JSON yang dilengkapi dengan beberapa parameter seperti: \textit{name}, \textit{type}, \textit{address} dan \textit{location} & Sistem menampilkan seluruh data perangkat yang terdaftar pada sistem. data yang ditampilkan meliputi: nama perangkat, tipe perangkat, alamat perangkat dan lokasi perangkat. serta terdapat tombol informasi untuk melihat data masing-masing perangkat secara rinci dan tombol hapus untuk menghapus data perangkat yang terdaftar pada sistem. \\ \hline
            	5 & /devices /edit /<string:id> & Mengakses endpoint dengan header autentikasi JWT, metode POST, menyertakan ID perangkat pada endpoint dan disertai dengan body bertipe JSON yang dilengkapi dengan beberapa parameter seperti: \textit{name}, \textit{type}, \textit{address} dan \textit{location}. & Sistem menghapus data perangkat terkait secara permanen. \\ \hline
            	6 & /devices /delete & Mengakses endpoint dengan header autentikasi JWT dan metode DELETE, disertai dengan body bertipe JSON yang dilengkapi dengan parameter ID perangkat. & Sistem mengubah data yang lama dengan data yang baru dimasukkan oleh pengguna. \\ \hline
            	7 & /oid /create & Mengakses endpoint dengan header autentikasi JWT dan metode POST, disertai dengan body bertipe JSON yang dilengkapi dengan beberapa parameter seperti: oidname, oid dan devices\_id & Sistem menandai bahwa perangkat atau OID (Informasi pada perangkat) yang terkait telat dilanggani. tombol akan berubah menjadi "Unsubscribe"\\ \hline
            	8 & /oid /edit & Mengakses endpoint dengan header autentikasi JWT dan metode POST, disertai dengan body bertipe JSON yang dilengkapi dengan beberapa parameter seperti: oidname, oid dan devices\_id & Sistem menampilkan kondisi dari seluruh perangkat yang telah dilanggani informasinya oleh pengguna. \\ \hline
            	9 & /oid /delete & Mengakses endpoint dengan header autentikasi JWT dan metode POST, disertai dengan body bertipe JSON yang dilengkapi dengan parameter parameter ID perangkat. & Sistem menampilkan kondisi dari seluruh perangkat yang telah dilanggani informasinya oleh pengguna. \\ \hline
            	10 & /subscribe /devices & Mengakses endpoint dengan header autentikasi JWT dan metode POST, disertai dengan body bertipe JSON yang dilengkapi dengan beberapa parameter seperti: device\_id dan users\_id. & Sistem menampilkan kondisi dari seluruh perangkat yang telah dilanggani informasinya oleh pengguna. \\ \hline
            	11 & /unsubscribe /devices & Mengakses endpoint dengan header autentikasi JWT dan metode POST, disertai dengan body bertipe JSON yang dilengkapi dengan beberapa parameter seperti: device\_id dan users\_id. & Sistem menampilkan kondisi dari seluruh perangkat yang telah dilanggani informasinya oleh pengguna. \\ \hline
            	12 & /subscribe /oid & Mengakses endpoint dengan header autentikasi JWT dan metode POST, disertai dengan body bertipe JSON yang dilengkapi dengan beberapa parameter seperti: device\_id dan users\_id. & Sistem menampilkan kondisi dari seluruh perangkat yang telah dilanggani informasinya oleh pengguna. \\ \hline
            	13 & /unsubscribe /oid & Mengakses endpoint dengan header autentikasi JWT dan metode POST, disertai dengan body bertipe JSON yang dilengkapi dengan beberapa parameter seperti: device\_id dan users\_id. & Sistem menampilkan kondisi dari seluruh perangkat yang telah dilanggani informasinya oleh pengguna. \\ \hline
            \end{longtable}
            
        
    \subsection{Skenario Uji Coba Performa}
    	Uji performa dilakukan dengan menggunakan lima buah desktop untuk melakukan akses secara bersamaan ke aplikasi menggunakan aplikasi JMeter. Desktop akan mencoba mengaskses halaman dari aplikasi web yang sudah berjalan, dengan domain aplikasi.nota-no.life. Halaman yang akan diakses berisi sebuah teks yang dihasilkan dari pemanggilan fungsi PHP. \\
        \indent Percobaan dilakukan dengan lima skenario jumlah \textit{concurrent user} yang berbeda, yaitu sebanyak 800, 1600, 2400, 3200, dan 4000 pengguna dalam rentang waktu inisialisasi $\pm$ 15 detik. Waktu tersebut menunjukkan masing-masing pengguna akan mengirimkan request selama $\pm$ 15 detik, namun tidak termasuk waktu menunggu balasan dari \textit{server}, yang artinya keseluruhan permintaan tersebut akan lebih dari waktu tersebut dan bergantung pada kemampuan \textit{server} untuk memberikan respon. Pengujian \textit{request} ini bertujuan untuk mengukur kemampuan dari \textit{proactive model}. Untuk masing-masingnya, dicoba sebanyak empat perhitungan \textit{proactive model} yang berbeda menggunakan ARIMA yang berbeda, yaitu ARIMA(1,1,0), ARIMA(2,1,0), ARIMA(3,1,0), ARIMA(4,1,0). \textit{Proactive model} sendiri berguna untuk mengetahui jumlah \textit{request} kedepannya agar sistem bisa menyediakan sumber daya berdasarkan predeksi tersebut. \\
        \indent Selain itu, untuk memperkirakan sumber daya yang dibutuhkan sistem kedepannya, digunakan \textit{reactive model}. \textit{Model} tersebut akan menghitung jumlah \textit{container} yang sumber daya CPU dan \textit{memory}-nya sudah melebihi batas yang ditentukan. Sistem akan membentuk \textit{container} baru berdasarkan perhitungan \textit{reactive model} tersebut jika ada \textit{container} yang penggunaannya sudah melebihi batas atas dan mengurangi \textit{container} jika ada \textit{container} yang tidak digunakan. Percobaan akan dilakukan sebanyak enam kali dan berikutnya akan dijelaskan data apa yang diuji untuk masing-masingnya. \\
        \indent
    	\subsubsection{Uji Performa Kecepatan Menangani \textit{Request}}
        	Pengujian dilakukan dengan mengukur jumlah waktu yang diperlukan untuk menyelesaikan \textit{request} yang dilakukan oleh komputer penguji. Waktu yang diukur adalah perbedaan jarak antara \textit{request} pertama dan yang terakhir dilakukan oleh klien yang mendapatkan balasan dari \textit{server}.
        \subsubsection{Uji Performa Penggunaan CPU}
        	Pengujian dilakukan dengan menghitung penggunaan CPU yang terjadi pada \textit{server master host}. Penggunaan CPU di sini adalah penggunaan dari \textit{container} aplikasi yang sedang berjalan. Perhitungan dilakukan dengan mengambil nilai rata-rata penggunaan CPU dari masing-masing \textit{container} selama proses pengujian dilakukan. Nilai yang didaptkan berupa total persen penggunaan CPU oleh \textit{container} dibandingkan dengan keseluruhan kemampuan CPU.
        \subsubsection{Uji Performa Penggunaan \textit{Memory}}
        	Pengujian dilakukan dengan menghitung penggunaan \textit{memory} yang terjadi pada \textit{server master host}. Penggunaan \textit{memory} di sini adalah penggunaan dari \textit{container} aplikasi yang sedang berjalan. Perhitungan dilakukan dengan mengambil nilai rata-rata pengguanaan \textit{memory} dari masing-masing aplikasi selama proses pengujian dilakukan.
        \subsubsection{Uji Performa Keberhasilan \textit{Request}}
        	Pengujian dilakukan dengan menghitung jumlah \textit{request} yang gagal dilakukan selama skenario dijalankan. Dari semua jumlah \textit{request} yang dikirimkan selama pengujian, akan didapatkan persen \textit{request} yang gagal dilakukan.
    
\section{Hasil Uji Coba dan Evaluasi}
	Berikut dijelaskan hasil uji coba dan evaluasi berdasarkan skenario yang telah dijelaskan pada subbab \ref{skenarioujicoba}.
    
	\subsection{Uji Fungsionalitas}
    	Berikut dijelaskan hasil pengujian fungsionalitas pada sistem yang dibangun.
        
        \subsubsection{Uji Mengelola Aplikasi Berbasis Docker}
    	Pengujian dilakukan sesuai dengan skenario yang dijelaskan pada subbab \ref{ujimengelolaaplikasiberbasisdocker} dan pada Tabel \ref{ujiaplikasi}. Hasil pengujian seperti tertera pada Tabel \ref{hasilujicobaaplikasi}.
        
        \begin{longtable}{|p{0.05\textwidth}|p{0.55\textwidth}|p{0.22\textwidth}|}					\caption{Hasil Uji Coba Mengelola Aplikasi Berbasis Docker} \label{hasilujicobaaplikasi} \\
					\hline
					\textbf{No} & \textbf{Uji Coba} & \textbf{Hasil} \\ \hline
					\endfirsthead
					\caption[]{Hasil Uji Coba Mengelola Aplikasi Berbasis Docker} \\
					\hline
					\textbf{No} & \textbf{Uji Coba} & \textbf{Hasil} \\ \hline
					\endhead
					\endfoot
					\endlastfoot
					
                    1 & Pengguna melakukan \textit{login} ke \textit{server docker registry} & OK. \\ \hline
                    2 & Pengguna menambahkan \textit{image} baru dari sebuah aplikasi ke \textit{server docker registry}. & OK. \\ \hline
                    3 & Pengguna bisa mengatur \textit{port} dari aplikasi menggunakan dasbor yang disediakan & OK. \\ \hline
                    4 & Pengguna bisa menjalankan aplikasi melalui fitur yang ada pada dasbor. & OK. \\ \hline
					5 & Pengguna memperbarui aplikasi yang sedang berjalan dengan melakukan \textit{push} ke \textit{server docker registry}. & OK. \\ \hline
                    6 & Penggua menghentikan aplikasi yang sedang berjalan. & OK. \\ \hline
				\end{longtable}
    		Sesuai dengan skenario uji coba  yang diberikan pada Tabel \ref{ujiaplikasi}, hasil uji coba menunjukkan semua skenario berhasil ditangani.
        
    	\subsubsection{Uji Fungsionalitas Menu Aplikasi Dasbor}
        	Sesuai dengan skenario pengujian yang dilakukan pada aplikasi dasbor. Pengujian dilakukan dengan menguji setiap menu pada aplikasi dasbor. Hasil uji coba dapat dilihat pada Table \ref{hasilUjiDasboard}. Semua skenario yang direncanakan berhasil ditangani.
        		\begin{longtable}{|p{0.05\textwidth}|p{0.20\textwidth}|p{0.30\textwidth}|p{0.27\textwidth}|}
						\caption{Hasil Uji Fungsionalitas Aplikasi Dasbor} \label{hasilUjiDasboard} \\
						\hline
						\textbf{No} & \textbf{Menu} & \textbf{Uji Coba} & \textbf{Hasil} \\ \hline
						\endfirsthead
						\caption[]{Hasil Uji Fungsionalitas Aplikasi Dasbor}  \\
						\hline
						\textbf{No} & \textbf{Menu} & \textbf{Uji Coba} & \textbf{Hasil} \\ \hline
						\endhead
						\endfoot
						\endlastfoot
						1 & Kelola aplikasi & Menambahkan aplikasi baru atau memperbarui aplikasi & Dasbor berhasil menampilkan daftar aplikasi terbaru yang dimasukkan atau diperbarui oleh pengembang. \\ \cline{3-4}
                        && Menjalankan aplikasi yang sudah masuk ke dalam sistem & Aplikasi berhasil berjalan dan pengguna mendapatkan domain untuk mengakses aplikasi. \\ \cline{3-4}
                        && Menghentikan aplikasi yang sedang berjalan & Aplikasi yang sedang berjalan berhasil dihentikan dan pengguna tidak bisa lagi melakukan akses terhadap aplikasi. \\ \cline{3-4}
                        && Mengganti \textit{port} aplikasi agar dapat berjalan dengan baik & Pengguna berhasil mengganti \textit{port} aplikasi agar aplikasi dapat berjalan dengan benar. \\ \hline
						2 & Lihat informasi aplikasi & Memilih salah satu aplikasi yang ada  & Pengguna berhasil melihat infromasi secara lengkap tentang aplikasi. \\ \hline
                        3 & Lihat informasi \textit{container} & Memilih salah satu aplikasi yang ada  & Pengguna berhasil melihat infromasi secara lengkap tentang \textit{container} yang sedang berjalan untuk aplikasi tersebut. \\ \hline
                        4 & Lihat metrik aplikasi & Memilih salah satu aplikasi yang ada  & Pengguna berhasil melihat grafik penggunaan CPU dan \textit{memory} dari aplikasi . \\ \hline
					\end{longtable}
    \pagebreak
    \subsection{Hasil Uji Performa}
    	Seperti yang sudah dijelaskan pada subbab \ref{skenarioujicoba} pengujian performa dilakukan dengan melakukan akses ke aplikasi dengan sejumlah pengguna secara bersama-sama. Pengujian dilaukan dengan memberikan \textit{request} secara berkelanjutan dengan jumlah pengguna terdiri dari lima bagian, yaitu 800, 1600, 2400, 3200, dan 4000 pengguna. Untuk jumlah \textit{request} yang dihasilkan dari masing-masing pengguna selama rentang waktu request $\pm$ 15 detik dapat dilihat pada Tabel \ref{trequest}. Jumlah tersebut akan diolah oleh \textit{reactive model}. Lalu jumlah penggunaan CPU dan memory selama menangani \textit{request} tersebut akan digunakan oleh \textit{proactive model} untuk menambahkan atau mengurangi \textit{container} yang ada.
        \begin{longtable}{|p{0.33\textwidth}|p{0.35\textwidth}|}
        \caption{Jumlah \textit{Request} ke Aplikasi} \label{trequest} \\
            \hline
            \textbf{\textit{Concurrent Users}} & \textbf{Jumlah \textit{Request}} \\ \hline
            \endfirsthead
            \caption[]{Jumlah \textit{Request} ke Aplikasi} \\
            \hline
            \textbf{\textit{Concurrent Users}} & \textbf{Jumlah \textit{Request}} \\ \hline
            \endhead
            \endfoot
            \endlastfoot
            
            800 & $\pm$ 16.925 \\ \hline
            1.600 & $\pm$ 26.650 \\ \hline
            2.400 & $\pm$ 34.943 \\ \hline
            3.200 & $\pm$ 50.092 \\ \hline
            4.000 & $\pm$ 57.750 \\ \hline
					
		\end{longtable}
        
        Pada Tabel \ref{tjumlahcontainer} dapat dilihat jumlah \textit{container} yang terbentuk selama proses \textit{request} dari \textit{user} yang dilakukan selama enam kali. Nilai yang ditampilkan berupa nilai rata-rata selama percobaan dibulatkan ke atas. Sistem dapat menyediakan \textit{container} sesuai dengan jumlah \textit{request} yang diberikan, semakin banyak \textit{request} yang dilakukan, maka \textit{container} yang disediakan akan semakin banyak. Nilai \textit{container} tersebut didapatkan dari perhitungan \textit{proactive model}. Selain melihat jumlah \textit{request}, penentuan \textit{container} yang dibentuk juga dari jumlah sumber daya yang digunakan \textit{container} berdasarkan perhitungan menggunakan \textit{reactive model}. Pada Gambar \ref{gjumlahcontainer} dapat dilihat grafik dari jumlah \textit{container} yang terbentuk berdasarkan jumlah \textit{request} yang dilakukan.
        
        \begin{longtable}{|p{0.25\textwidth}|p{0.20\textwidth}|p{0.20\textwidth}|}
        \caption{Jumlah \textit{Container}} \label{tjumlahcontainer} \\
            \hline
            \textbf{\textit{Concurrent Users}} & \textbf{Maksimal \textit{Container}} &  \textbf{Rata-rata \textit{Container}} \\ \hline
            \endfirsthead
            \caption[]{Jumlah \textit{Container}} \\
            \hline
            \textbf{\textit{Concurrent Users}} & \textbf{Maksimal \textit{Container}} &  \textbf{Rata-rata \textit{Container}} \\ \hline
            \endhead
            \endfoot
            \endlastfoot
            
            800 & 6 & 2 \\ \hline
            1.600 & 8 & 3 \\ \hline
            2.400 & 14 & 6 \\ \hline
            3.200 & 18 & 7 \\ \hline
            4.000 & 30 & 11 \\ \hline
					
		\end{longtable}
        
        \begin{figure}[H]
				\centering
				\includegraphics[width=8.7cm,height=4.7cm]{Images/C-5/jumlahcontainer.png}
				\caption{Grafik Jumlah \textit{Container}}
				\label{gjumlahcontainer}
			\end{figure}
        
    	\subsubsection{Kecepatan Menangani \textit{Request}}
        	Dari hasil uji coba kecepatan menangani \textit{request}, dapat dilihat pada Table \ref{kecepatanrequest} dalam satuan detik bahwa semakin banyak \textit{concurrent users}, semakin lama pula waktu yang diperlukan untuk menyeselaikannya. Request paling cepat ditangani dengan menggunakan prediksi ARIMA(4,1,0) dan paling lambat menggunakan ARIMA(1,1,0). Hal tersebut terjadi karena kurang bagusnya hasil prediksi yang dihasilkan oleh ARIMA(1,1,0) yang mana kadang hasil prediksinya terlalu rendah atau terlalu tinggi.
            Dari hasil percobaan tersebut, dapat dilihat bahwa hampir semua \textit{request} dapat ditangani di bawah satu menit. Lalu grafik hasil uji coba perhitungan kecepatan menangani \textit{request} ditunjukkan pada Gambar \ref{grunningtime}.
            \begin{longtable}{|p{0.22\textwidth}|p{0.10\textwidth}|p{0.10\textwidth}|p{0.10\textwidth}|p{0.10\textwidth}|p{0.10\textwidth}|}
        \caption{Kecepatan Menangani \textit{Request}} \label{kecepatanrequest} \\
            \hline
            & \textbf{800} & \textbf{1600} & \textbf{2400} & \textbf{3200} & \textbf{4000} \\ \hline
            \endfirsthead
            \caption[]{Kecepatan Menangani \textit{Request}} \\
            \hline
            & \textbf{800} & \textbf{1600} & \textbf{2400} & \textbf{3200} & \textbf{4000} \\ \hline
            \endhead
            \endfoot
            \endlastfoot
			
          	ARIMA(1,1,0) & 34.167 & 43.286 & 48.143 & 63.857 & 62.286 \\ \hline
            ARIMA(2,1,0) & 27.429 & 38.571 & 44.143 & 42.143 & 57.857 \\ \hline
            ARIMA(3,1,0) & 32.429 & 36.000 & 38.429 & 41.571 & 43.857 \\ \hline
            ARIMA(4,1,0) & 24.857 & 31.571 & 34.429 & 42.143 & 52.714 \\ \hline
		\end{longtable}
         
        	\begin{figure}[H]
				\centering
				\includegraphics[width=8.7cm,height=4.7cm]{Images/C-5/runningtime.png}
				\caption{Grafik Kecepatan Menangani \textit{Request}}
				\label{grunningtime}
			\end{figure}
            
        \subsubsection{Penggunaan CPU}
        	Dari hasil uji coba penggunaan CPU pada \textit{server master host}, penggunaan CPU berada di bawah 15\%. Penggunaan CPU yang diukur adalah penggunaan CPU yang dilakukan oleh \textit{container} dari aplikasi, tidak termasuk sistem. Jumlah \textit{core} yang dimiliki oleh \textit{processor} di \textit{server master host} adalah 8 buah, yang artinya kurang lebih hanya satu core yang digunakan untuk menangani semua \textit{request}. Hasil pengukuran penggunaan CPU dapat dilihat pada Tabel \ref{penggunaancpu}
            
            \begin{longtable}{|p{0.22\textwidth}|p{0.10\textwidth}|p{0.10\textwidth}|p{0.10\textwidth}|p{0.10\textwidth}|p{0.10\textwidth}|}
        \caption{Penggunaan CPU} \label{penggunaancpu} \\
            \hline
            & \textbf{800} & \textbf{1600} & \textbf{2400} & \textbf{3200} & \textbf{4000} \\ \hline
            \endfirsthead
            \caption[]{Penggunaan CPU} \\
            \hline
            & \textbf{800} & \textbf{1600} & \textbf{2400} & \textbf{3200} & \textbf{4000} \\ \hline
            \endhead
            \endfoot
            \endlastfoot
			
            ARIMA(1,1,0) & 7.1\% & 7.8\% & 9.1\% & 10.5\% & 10.7\% \\ \hline
            ARIMA(2,1,0) & 8.5\% & 9.2\% & 10.1\% & 11.3\% & 10.7\% \\ \hline
            ARIMA(3,1,0) & 8.8\% & 10.2\% & 11.6\% & 12.1\% & 10.3\% \\ \hline
            ARIMA(4,1,0) & 8.0\% & 8.3\% & 10.1\% & 12.9\% & 10.5\% \\ \hline

		\end{longtable}
            
            Dari hasil uji coba, penggunaan prediksi yang berbeda tidak terlalu berpengaruh terhadap penggunaan CPU. Lalu, penggunaan CPU tergolong rendah, yaitu hanya sebesar $\pm 10 \%$ untuk menangani semua \textit{request} yang diberikan. Hasil uji coba performa penggunaan CPU ditunjukkan oleh dalam grafik pada Gambar \ref{gcpuusage}.
            
        	\begin{figure}[H]
				\centering
				\includegraphics[width=8.7cm,height=4.7cm]{Images/C-5/cpuusage.png}
				\caption{Grafik Penggunaan CPU}
				\label{gcpuusage}
			\end{figure}
            
        \subsubsection{Penggunaan \textit{Memory}}            
            Dari hasil uji coba penggunaan \textit{memory}, semakin banyak \textit{request} yang diterima, semakin banyak \textit{memory} yang diperlukan. Perhitungan penggunaan \textit{memory} adalah rata-rata penggunaan dari masing-masing \textit{container} sebuah aplikasi. Untuk masing-masing \textit{container}, dibatasi penggunaan maksimal \textit{memory} adalah 512 MB. Dari hasil uji coba ini, dapat dilihat pada Tabel \ref{penggunaanmemory} bahwa penggunaan terbesar hanya sebesar 158.71 MB. Artinya jumlah tersebut hanya menggunakan sepertiga dari keseluruhan \textit{memory} yang bisa digunakan.
            \begin{longtable}{|p{0.22\textwidth}|p{0.10\textwidth}|p{0.10\textwidth}|p{0.10\textwidth}|p{0.10\textwidth}|p{0.10\textwidth}|}
        \caption{Penggunaan \textit{Memory}} \label{penggunaanmemory} \\
            \hline
            & \textbf{800} & \textbf{1600} & \textbf{2400} & \textbf{3200} & \textbf{4000} \\ \hline
            \endfirsthead
            \caption[]{Penggunaan \textit{Memory}} \\
            \hline
            & \textbf{800} & \textbf{1600} & \textbf{2400} & \textbf{3200} & \textbf{4000} \\ \hline
            \endhead
            \endfoot
            \endlastfoot
			
           	ARIMA(1,1,0) & 67.91 & 88.97 & 130.79 & 120.14 & 157.73 \\ \hline
            ARIMA(2,1,0) & 65.89 & 97.98 & 123.47 & 156.64 & 158.33 \\ \hline
            ARIMA(3,1,0) & 72.20 & 99.72 & 125.56 & 144.42 & 152.14 \\ \hline
            ARIMA(4,1,0) & 69.60 & 77.34 & 117.39 & 149.76 & 158.71 \\ \hline

		\end{longtable}
            
            Hasil uji coba performa penggunaan \textit{memory} dalam grafik ditunjukkan pada Gambar \ref{gmemoryusage}.
            
        	\begin{figure}[H]
				\centering
				\includegraphics[width=8.7cm,height=4.7cm]{Images/C-5/memoryusage.png}
				\caption{Grafik Penggunaan Memory}
				\label{gmemoryusage}
			\end{figure}
            
        \subsubsection{Keberhasilan \textit{Request}}
        	Pada uji coba ini, dilakukan perhitungan seberapa besar jumlah \textit{request} yang gagal dilakukan. Untuk jumlah \textit{concurrent user} pada tingkat 800 dan 1600, dapat dilihat pada Table \ref{keberhasilanrequest} \textit{error} yang terjadi hampir sama. Prediksi menggunakan ARIMA(4,1,0) berhasil unggul karena menggunakan parameter yang lebih banyak. Namun hal tersebut tidak berlaku untuk ARIMA(3,1,0) karena walaupun parameternya lebih banyak dari ARIMA(2,1,0), tapi hasil prediksinya bisa meleset saat terjadi kondisi dimana koefisien negatif atau koefisien ke dua dikalikan dengan sebuah parameter bukan nol, dan koefisien lain dikalikan dengan parameter nol, maka hasil prediksinya akan negatif, yang mana seharusnya tidak mungkin ada \textit{request} negatif.
            \begin{longtable}{|p{0.22\textwidth}|p{0.10\textwidth}|p{0.10\textwidth}|p{0.10\textwidth}|p{0.10\textwidth}|p{0.10\textwidth}|}
        \caption{\textit{Error Ratio Request}} \label{keberhasilanrequest} \\
            \hline
            & \textbf{800} & \textbf{1600} & \textbf{2400} & \textbf{3200} & \textbf{4000} \\ \hline
            \endfirsthead
            \caption[]{\textit{Error Ratio Request}} \\
            \hline
            & \textbf{800} & \textbf{1600} & \textbf{2400} & \textbf{3200} & \textbf{4000} \\ \hline
            \endhead
            \endfoot
            \endlastfoot
			
            ARIMA(1,1,0) & 5.72\% & 8.96\% & 12.85\% & 12.54\% & 13.38\% \\ \hline
            ARIMA(2,1,0) & 4.31\% & 9.35\% & 10.68\% & 8.11\% & 9.04\% \\ \hline
            ARIMA(3,1,0) & 4.84\% & 10.02\% & 13.22\% & 8.63\% & 12.24\% \\ \hline
            ARIMA(4,1,0) & 4.62\% & 8.41\% & 9.39\% & 7.52\% & 9.21\% \\ \hline
		\end{longtable}
            
    		\begin{figure}[H]
				\centering
				\includegraphics[width=8.7cm,height=4.4cm]{Images/C-5/errorratio.png}
				\caption{Grafik Error Ratio}
				\label{gerrorratio}
			\end{figure}
            Dari uji coba itu, 90\% lebih \textit{request} berhasil ditangani. Hasil uji coba jumlah \textit{request} yang gagal ditunjukkan dengan grafik pada Gambar \ref{gerrorratio}.
	\chapter{PENUTUP}
    Bab ini membahas kesimpulan yang dapat diambil dari tujuan pembuatan sistem dan hubungannya dengan hasil uji coba dan evaluasi yang telah dilakukan. Selain itu, terdapat beberapa saran yang bisa dijadikan acuan untuk melakukan pengembangan dan penelitian lebih lanjut.
        
	\section{Kesimpulan}
        Dari proses perencangan, implementasi dan pengujian terhadap sistem, dapat diambil beberapa kesimpulan berikut:
		\begin{enumerate}
            \item Sistem dapat menjalankan dan menyajikan satu atau lebih aplikasi web berbasis \textit{docker} kepada \textit{end-user} melalui domain yang disediakan.
            \item Sistem dapat menyesuaikan sumber daya secara otomatis berdasarkan jumlah \textit{request} dengan menggunakan \textit{proactive model} dan penggunaan sumber daya, yaitu CPU dan \textit{memory}, pada \textit{container} dengan menggunakan \textit{reactive model}.
            \item Penggunaan \textit{load balancer} cocok digunakan dengan aplikasi yang berjalan di atas \textit{docker} \textit{container}. Hal tersebut karena semua \textit{request} ke aplikasi akan melalui \textit{load balancer}. Jika terjadi penambahan dan pengurangan sumber daya, penyesuaian dengan cepat dilakukan dan hanya perlu merubah sedikit konfigurasi pada \textit{load balancer} dan pengguna tidak perlu tahu apa yang terjadi di dalam sistem.
            \item Prediksi jumlah \textit{request} menggunakan ARIMA sudah bisa menangani skenario uji coba. Perbedaan \textit{order} ARIMA yang digunakan mempengaruhi akurasi dalam menentukan \textit{request} yang akan terjadi ke depannya. Dalam pengujian ini, ARIMA(4,1,0) memiliki hasil pengujian paling bagus dengan jumlah rata-rata \textit{error request} yang paling rendah, yaitu sebesar 7.83\%. Lalu untuk kecepatan menerima \textit{request}, ARIMA(2,1,0) dan ARIMA(4,1,0) memiliki konsistensi yang berbanding lurus dengan jumlah \textit{request}.
            \item Penggunaan sumber daya CPU dan \textit{memory} tidak dipengaruhi oleh penggunaan ARIMA yang berbeda. Penggunaan sumber daya tersebut bergantung kepada jumlah \textit{request}, semakin banyak \textit{request} yang diberikan, penggunaan CPU dan \textit{memory} akan semakin tinggi. Penggunaan CPU paling tinggi yaitu sebesar 12.9\% dan penggunaan \textit{memory} paling tinggi sebesar 158.71 MB. Dengan penggunaan tersebut, masih tersisa lebih dari setengah sumber daya yang bisa digunakan.
            \item Sebuah \textit{container} dari sebuah aplikasi dapat dibentuk dalam waktu $\pm$ 1 detik sehingga penambahan sumber daya bisa dilakukan dengan cepat dan proses untuk memperbarui konfigurasi dari HAProxy memerlukan waktu $\pm$ 5 detik. Selama proses tersebut, akses pengguna akan tertunda, namun tidak menunjukkan terjadinya \textit{down}.
		\end{enumerate}
        
	\section{Saran}
		Berikut beberapa saran yang diberikan untuk pengembangan lebih lanjut:
		\begin{enumerate}
			\item Mengamankan komunikasi antar \textit{server} karena saat ini \textit{endpoint server} bisa diakses oleh siapapun. Hal tersebut bisa dilakukan dengan mengimplentasikan \textit{private} IP dan menggunakan token untuk komunikasinya.
            \item Pemodelan menggunakan ARIMA cukup baik, namun perlu dicoba untuk melakukan pembuatan model dengan \textit{dataset} yang lebih baru. Selain itu, bisa mencoba alternatif pemodelan \textit{time series} yang lain, seperti ARCH (Autoregressive Conditional Heteroskedasticity).
		\end{enumerate}

	\bibliography{Zotero}
	\bibliographystyle{IEEEtranID.bst}
    
    \renewcommand\chaptername{LAMPIRAN}
	\appendix
    \chapter{INSTALASI PERANGKAT LUNAK}

\section*{Instalasi Lingkungan Docker}
	Proses pemasangan Docker dpat dilakukan sesuai tahap berikut:
    \begin{itemize}
    \item Menambahkan repository Docker\\
    	Langkah ini dilakukan untuk menambahkan \textit{repository} Docker ke dalam paket \texttt{apt} agar dapat di unduh oleh Ubuntu. Untuk melakukannya, jalankan perintah berikut:
		\begin{tabbing}
          \texttt{sudo apt-get -y install \char`\\} \\
          \hspace{5 mm} \texttt{apt-transport-https \char`\\} \\
          \hspace{5 mm} \texttt{ca-certificates \char`\\} \\
          \hspace{5 mm} \texttt{curl} \\
          \\
          \texttt{curl -fsSL https://download.docker.com/linux/} \\
          \hspace{7 mm} \texttt{ubuntu/gpg | sudo apt-key add -} \\
          \\
          \texttt{sudo add-apt-repository \char`\\} \\
          \hspace{7 mm} \texttt{"deb [arch=amd64] https://download.docker.com/} \\
          \hspace{9 mm} \texttt{linux/ubuntu \char`\\} \\
          \hspace{7 mm} \texttt{\$ (lsb\_release -cs) \char`\\} \\
          \hspace{7 mm} \texttt{stable"} \\
          \\
          \texttt{sudo apt-get update} \\
        \end{tabbing}
        
    \item Mengunduh Docker \\
    	Docker dikembangkan dalam dua versi, yaitu CE (\textit{Community Edition}) dan EE (\textit{Enterprise Edition}). Dalam pengembangan sistem ini, digunakan Docker CE karena merupakan versi Docker yang gratis. Untuk mengunduh Docker CE, jalankan perintah \texttt{sudo apt-get -y install docker-ce}.
    
    \item Mencoba menjalankan Docker \\
    	Untuk melakukan tes apakah Docker sudah terpasang dengan benar, gunakan perintah \texttt{sudo docker run hello-world}.
    \end{itemize}

\section*{Instalasi Docker Registry} \label{install:dockerRegistry}
	Docker Registry dikembangkan menggunakan Docker Compose. Dengan menggunakan Docker Compose, proses pemasangan Docker Registry menjadi lebih mudah dan fleksibel untuk dikembangkan ditempat lain. Docker Registry akan dijalankan pada satu \textit{container} dan Nginx juga akan dijalankan di satu \textit{container} lain yang berfungsi sebagai perantara komunikasi antara Docker Registry dengna dunia luar. Berikut adalah proses pengembangan Docker Registry yang penulis lakukan:
    \begin{itemize}
    \item Pemasangan Docker Compose\\
    \$ \texttt{sudo apt-get -y install python-pip} \\
    \$ \texttt{sudo pip install docker-compose}
    
    \item Pemasangan paket \texttt{apache2-utils}\\
    	Pada paket \texttt{apache2-utils} terdapat fungsi \texttt{htpasswd} yang digunakan untuk membuat \textit{hash password} untuk Nginx. Proses pemasangan paket dapat dilakukan dengan menjalankan perintah \texttt{sudo apt-get -y install apache2-utils}.
        
    \item Pemasangan dan pengaturan Docker Registry\\
    	Buat folder \texttt{docker-registry} dan \texttt{data} dengan menjalankan perintah berikut:\\
        \$ \texttt{mkdir ~/docker-registry \&\& cd \$\_} \\
        \$ \texttt{mkdir data} \\
        Folder \texttt{data} digunakan untuk menyimpan data yang dihasilkan dan digunakan oleh \textit{container} Docker Registry. Kemudian di dalam folder \texttt{docker-registry} buat sebuah berkas dengan nama \texttt{docker-compose.yml} yang akan digunakan oleh Docker Compose untuk membangun aplikasi. Tambahkan isi berkasnya sesuai dengan Kode Sumber \ref{dockerCompose}.
        
        \begin{lstlisting}[frame=single,tabsize=2,breaklines,caption={Isi Berkas docker-compose.yml},label=dockerCompose, captionpos=b]
nginx:
image: "nginx:1.9"
ports:
	- 443:443
	- 80:80
links:
	- registry:registry
volumes:
	- ./nginx/:/etc/nginx/conf.d
registry:
	image: registry:2
	ports:
		- 127.0.0.1:5000:5000
	environment:
		REGISTRY_STORAGE_FILESYSTEM _ROOTDIRECTORY: /data
	volumes:
		- ./data:/data
		- ./registry/config.yml:/etc/docker/registry/config.yml
		\end{lstlisting}
	
    \item Pemasangan \textit{container} Nginx
    	Buat folder \texttt{nginx} di dalam folder \texttt{docker-registry}. Di dalam folder \texttt{nginx} buat berkas dengan nama \texttt{registry.conf} yang berfungsi sebagai berkas konfigurasi yang akan digunakan oleh Nginx. Isi berkas sesuai denga Kode Sumber \ref{registryConf}.
        \begin{lstlisting}[frame=single,tabsize=2,breaklines,caption={Isi Berkas registry.conf},label=registryConf, captionpos=b]
upstream docker-registry{
  server registry:5000;
}
server{
  listen 80;
  server_name registry.nota-no.life;
  return 301 https://$server_name$request_uri;
}
server{
  listen 443;
  server_name registry.nota-no.life;
  ssl on;
  ssl_certificate /etc/nginx/conf.d/cert.pem;
  ssl_certificate_key /etc/nginx/conf.d/privkey.pem;
  client_max_body_size 0;
  chunked_transfer_encoding on;
  location /v2/{
    if ($http_user_agent ~ "^(docker\/1\.(3|4|5(?!\.[0-9]-dev))|Go ).*$" ){
      return 404;
    }
    auth_basic "registry.localhost";
    auth_basic_user_file /etc/nginx/conf.d/registry.password;
    add_header 'Docker-Distribution-Api-Version' 'registry/2.0' always;
    proxy_pass http://docker-registry;
    proxy_set_header Host $http_host;
    proxy_set_header X-Real-IP $remote_addr;
    proxy_set_header X-Forwarded-For $proxy_add_x_forwarded_for;
    proxy_set_header X-Forwarded-Proto $scheme;
    proxy_read_timeout 900;
  }
}
		\end{lstlisting}
        
    \end{itemize}

\section*{Instalasi Pustaka Python} \label{install:pythonlibrary}
	Dalam pengembangan sistem ini, digunakan berbagai pustaka pendukung. Pustaka pendukung yang digunakan merupakan pustaka untuk bahasa pemrograman Python. Berikut adalah daftar pustaka yang digunakan dan cara pemasangannya:
    \begin{itemize}
    \item Python Dev \\
    	\$ \texttt{sudo apt-get install python-dev}
    \item Flask \\
    	\$ \texttt{sudo pip install Flask}
    \item docker-py \\
    	\$ \texttt{sudo pip install docker}
    \item MySQLd \\
    	\$ \texttt{sudo apt-get install python-mysqldb}
    \item Redis \\
    	\$ \texttt{sudo pip install redis}
    \item RQ \\
    	\$ \texttt{sudo pip install rq}
    \end{itemize}

\section*{Instalasi HAProxy} \label{install:haproxy}
	HAProxy dapat dipasang dengna mudah menggunakan \texttt{apt-get} karena perangkat lunak tersebut sudah tersedia pada \textit{repository} Ubuntu. Untuk melakukan pemasangan HAProxy, gunakan perintah \texttt{apt-get install haproxy}. \\
    \indent Setelah HAProxy diunduh, perangkat lunak tersebut belum berjalan karena belum diaktifkan. Untuk mengaktifkan \textit{service haproxy}, buka berkas di \texttt{/etc/default/harpoxy} kemudian ganti nilai \texttt{ENABLED} yang awalnya bernilai \texttt{0} menjadi \texttt{ENABLED=1}. Setelah itu service haproxy dapat dijalankan dengan menggunakan perintah \texttt{service harpoxy start}.
    \indent Untuk konfigurasi dari HAProxy nantinya akan diurus oleh \textit{confd}. \textit{confd} akan menyesuaikan konfigurasi dari HAProxy sesuai dengan kebutuhan aplikasi yang tersedia.

\section*{Instalasi etcd dan confd} \label{install:etcdconfd}
	etcd dapat di unggah dengan menjalankan perintah berikut, \texttt{curl https://github.com/coreos/etcd/releases/ download/v3.2.0-rc.0/etcd-v3.2.0-rc.0-linux- amd64.tar.gz}. Setelah proses unduh berhasil dilakukan, selanjutnya yang dilakukan adalah melakukan ekstrak berkasnya menggunakan perintah \texttt{tar -xvzf etcd-v3.2.0-rc.0- linux-amd64.tar.gz}. Berkas binary dari etcd bisa ditemukan pada folder \texttt{./bin/etcd}. Berkas inilah yang digunakan untuk menjalankan perangkat lunak etcd. Untuk menjalankannya, dapat dilakukan dengan menggunakan perintah \texttt{etcd --listen-client-urls http://0.0.0.0:5050 --advertise-client-urls http://128.199.250.137 :5050}. Perintah tersebut memungkinkan etcd diakses oleh \textit{host} lain dengan IP 128.199.250.137, yang merupakan host dari \textit{load balancer} dan confd. Setelah proses tersebut, etcd sudah siap untuk digunakan. \\
    \indent Setelah etcd siap digunakan, selanjutnya adalah memasang confd. Untuk menginstall confd gunakan rangkaian perintah berikut: \\
    \$ \texttt{mkdir -p \$GOPATH/src/github.com/kelseyhightower} \\
	\$ \texttt{git clone https://github.com/kelseyhightower/ confd.git \$GOPATH/src/github.com/kelseyhightower/ confd} \\
	\$ \texttt{cd \$GOPATH/src/github.com/kelseyhightower/confd} \\
	\$ \texttt{./build}
	
	Setelah berhasil memasang confd, selanjutnya buka berkas \texttt{/etc/confd/confd.toml} dan isi berkas sesuai dengan Kode Sumber \ref{confdToml}. Pengaturan tersebut bertujuan agar confd melakukan \textit{listen} terhadap server etcd dan melakukan tindakan jika terjadi perubahan pada etcd.
	\begin{lstlisting}[frame=single,tabsize=2,breaklines,caption={Isi Berkas confd.toml},label=confdToml, captionpos=b]
confdir = "/etc/confd"
interval = 20
backend = "etcd"
nodes = [
        "http://128.199.250.137:5050"
]
prefix = "/"
scheme = "http"
verbose = true
		\end{lstlisting}
	Setelah melakukan konfigurasi confd, selanjutnya adalah membuat \textit{template} konfigurasi untuk HAProxy. Buka berkas di \texttt{/etc/confd/templates/haproxy.cfg.tmpl}. Jika berkas tidak ada maka buat berkasnya dan isi berkas sesuai dengan Kode Sumber \ref{haproxyCfgTmpl}.
        
    \begin{lstlisting}[frame=single,tabsize=2,breaklines,caption={Isi Berkas haproxy.cfg.tmpl},label=haproxyCfgTmpl, captionpos=b]
global
        log /dev/log    local0
        log /dev/log    local1 notice
        chroot /var/lib/haproxy
        stats socket /run/haproxy/admin.sock mode 660 level admin
        stats timeout 30s
        daemon
defaults
        log     global
        mode    http
        option  httplog
        option  dontlognull
        timeout connect 5000
        timeout client  50000
        timeout server  50000
        errorfile 400 /etc/haproxy/errors/400.http
        errorfile 403 /etc/haproxy/errors/403.http
        errorfile 408 /etc/haproxy/errors/408.http
        errorfile 500 /etc/haproxy/errors/500.http
        errorfile 502 /etc/haproxy/errors/502.http
        errorfile 503 /etc/haproxy/errors/503.http
        errorfile 504 /etc/haproxy/errors/504.http
frontend http-in
        bind *:80

        # Define hosts
        {{range gets "/images/*"}}
        {{$data := json .Value}}
                acl host_{{$data.image_name}} hdr(host) -i {{$data.domain}}.nota-no.life
        {{end}}

        ## Figure out which one to use
        {{range gets "/images/*"}}
        {{$data := json .Value}}
                use_backend {{$data.image_name}}_cluster if host_{{$data.image_name}}
        {{end}}
{{range gets "/images/*"}}
{{$data := json .Value}}
backend {{$data.image_name}}_cluster
        mode http
        balance roundrobin
        option forwardfor
        cookie JSESSIONID prefix
        {{range $data.containers}}
        server {{.name}} {{.ip}}:{{.port}} check
        {{end}}
{{end}}
		\end{lstlisting}    

	Langkah terakhir adalah membuat berkas konfigurasi untuk HAProxy di \texttt{/etc/confd/conf.d/haproxy.toml}. Jika berkas tidak ada, maka buat berkasnya dan isi berkas sesuai dengan Kode Sumber \ref{haproxyToml}.
	\begin{lstlisting}[frame=single,tabsize=2,breaklines,caption={Isi Berkas haproxy.toml},label=haproxyToml, captionpos=b]
[template]
src = "haproxy.cfg.tmpl"
dest = "/etc/haproxy/haproxy.cfg"
keys = [
        "/images"
]
reload_cmd = "iptables -I INPUT -p tcp --dport 80 --syn -j DROP && sleep 1 && service haproxy restart && iptables -D INPUT -p tcp --dport 80 --syn -j DROP"
	\end{lstlisting}
    
    Setelah melakukan konfigurasi, selanjutnya adalah menjalankan confd dengan menggunakan perintah \texttt{confd \&}.

\section*{Pemasangan Redis} \label{install:redis}
	Redis dapat dipasang dengan mempersiapkan kebutuhan pustaka pendukungnya. Pustaka yang digunakan adalah \texttt{build-essential} dan \texttt{tcl8.5}. Untuk melakukan pemasangannya, jalankan perintah berikut:\\
	 \$ \texttt{sudo apt-get install build-essential}\\
     \$ \texttt{sudo apt-get install tcl8.5}\\
     \indent Setelah itu unduh aplikasi Redis dengan menjalankan perintah \texttt{wget http://download.redis.io/releases/redis-stable.tar.gz}. Setelah selesai diunduh, buka file dengan perintah berikut:\\
     \$ \texttt{tar xzf redis-stable.tar.gz \&\& cd redis-stable}\\
     \indent Di dalam folder \texttt{redis-stable}, bangun Redis dari kode sumber dengan menjalankan perintah \texttt{make}. Setelah itu lakukan tes kode sumber dengan menjalankan \texttt{make test}. Setelah selesai, pasang Redis dengan menggunakan perinah \texttt{sudo make install}. Setelah selesai melakukan pemasangan, Redis dapat diaktifkan dengan menjalankan berkas bash dengan nama \texttt{install\_server.sh}.\\
     Untuk menambah pengaman pada Redis, diatur agar Redis hanya bisa dari \textit{localhost}. Untuk melakukannya, buka file \texttt{/etc/redis/6379.conf}, kemudian cari baris \texttt{bind 127.0.0.1}. Hapus komen jika sebelumnya baris tersebut dalam keadaan tidak aktif. Jika tidak ditemukan baris dengan isi tersebut, tambahkan pada akhir berkas baris tersebut.

\section*{Pemasangan kerangka kerja React}
	Pada pengembangan sistem ini, penggunaan pustaka React dibangun di atas konfigurasi Create React App. Untuk memasang Create React App, gunakan perintah \texttt{npm install -g create-react-app}. Setelah terpasang, untuk membangun aplikasinya jalankan perintah \texttt{create-react-app fe-controller}. Setelah proses tersebut, dasar dari aplikasi sudah terbangun dan siap untuk dikembangkan lebih lanjut.
    \chapter{KODE SUMBER}

\section*{Let's Encrypt Cross Signed}
\begin{lstlisting}[frame=single,tabsize=2,breaklines,caption={Let's Encrypt X3 Cross Signed.pem},label=letsencryptpem, captionpos=b]
-----BEGIN CERTIFICATE-----
MIIEkjCCA3qgAwIBAgIQCgFBQgAAAVOF c2oLheynCDANBgkqhkiG9w0BAQsFADA/
MSQwIgYDVQQKExtEaWdpdGFsIFNpZ25h dHVyZSBUcnVzdCBDby4xFzAVBgNVBAMT
DkRTVCBSb290IENBIFgzMB4XDTE2MDMx NzE2NDA0NloXDTIxMDMxNzE2NDA0Nlow
SjELMAkGA1UEBhMCVVMxFjAUBgNVBAoT DUxldCdzIEVuY3J5cHQxIzAhBgNVBAMT
GkxldCdzIEVuY3J5cHQgQXV0aG9yaXR5 IFgzMIIBIjANBgkqhkiG9w0BAQEFAAOC
AQ8AMIIBCgKCAQEAnNMM8FrlLke3cl03 g7NoYzDq1zUmGSXhvb418XCSL7e4S0EF
q6meNQhY7LEqxGiHC6PjdeTm86dicbp5 gWAf15Gan/PQeGdxyGkOlZHP/uaZ6WA8
SMx+yk13EiSdRxta67nsHjcAHJyse6cF 6s5K671B5TaYucv9bTyWaN8jKkKQDIZ0
Z8h/pZq4UmEUEz9l6YKHy9v6Dlb2honz hT+Xhq+w3Brvaw2VFn3EK6BlspkENnWA
a6xK8xuQSXgvopZPKiAlKQTGdMDQMc2P MTiVFrqoM7hD8bEfwzB/onkxEz0tNvjj
/PIzark5McWvxI0NHWQWM6r6hCm21AvA 2H3DkwIDAQABo4IBfTCCAXkwEgYDVR0T
AQH/BAgwBgEB/wIBADAOBgNVHQ8BAf8E BAMCAYYwfwYIKwYBBQUHAQEEczBxMDIG
CCsGAQUFBzABhiZodHRwOi8vaXNyZy50 cnVzdGlkLm9jc3AuaWRlbnRydXN0LmNv
bTA7BggrBgEFBQcwAoYvaHR0cDovL2Fw cHMuaWRlbnRydXN0LmNvbS9yb290cy9k
c3Ryb290Y2F4My5wN2MwHwYDVR0jBBgw FoAUxKexpHsscfrb4UuQdf/EFWCFiRAw
VAYDVR0gBE0wSzAIBgZngQwBAgEwPwYL KwYBBAGC3xMBAQEwMDAuBggrBgEFBQcC
ARYiaHR0cDovL2Nwcy5yb290LXgxLmxl dHNlbmNyeXB0Lm9yZzA8BgNVHR8ENTAz
MDGgL6AthitodHRwOi8vY3JsLmlkZW50 cnVzdC5jb20vRFNUUk9PVENBWDNDUkwu
Y3JsMB0GA1UdDgQWBBSoSmpjBH3duubR ObemRWXv86jsoTANBgkqhkiG9w0BAQsF
AAOCAQEA3TPXEfNjWDjdGBX7CVW+dla5 cEilaUcne8IkCJLxWh9KEik3JHRRHGJo
uM2VcGfl96S8TihRzZvoroed6ti6WqEB mtzw3Wodatg+VyOeph4EYpr/1wXKtx8/
wApIvJSwtmVi4MFU5aMqrSDE6ea73Mj2 tcMyo5jMd6jmeWUHK8so/joWUoHOUgwu
X4Po1QYz+3dszkDqMp4fklxBwXRsW10K XzPMTZ+sOPAveyxindmjkW8lGy+QsRlG
PfZ+G6Z6h7mjem0Y+iWlkYcV4PIWL1iw Bi8saCbGS5jN2p8M+X+Q7UNKEkROb3N6
KOqkqm57TH2H3eDJAkSnh6/DNFu0Qg==
-----END CERTIFICATE-----
	\end{lstlisting}


	\appendix

	\backmatter % Lampiran tanpa judul LAMPIRAN X, biasanya untuk BIODATA PENULIS
	\chapter{BIODATA PENULIS}
		\begin{wrapfigure}{l}{0.3\textwidth}
			\includegraphics[width=0.29\textwidth]{Images/MFR.jpg}
		\end{wrapfigure}
		
		\textbf{Muhammad Fahrul Razi}, akbrab dipanggil Razi lahir pada tanggal 23 November 1994 di Ilung, Kalimantan Selatan. Penulis merupakan seorang mahasiswa yang sedang menempuh studi di Jurusan Teknik Informatika Institut Teknologi Sepuluh Nopember. Memiliki hobi antara lain membaca novel dan futsal. Selama menempuh pendidikan di kampus, penulis juga aktif dalam organisasi kemahasiswaan, antara lain Staff Departemen Pengembangan Sumber Daya Mahasiswa Himpunan Mahasiswa Teknik Computer-Informatika pada tahun ke-2. Pernah menjadi staff National Programming Contest Schematics tahun 2014 dan 2015. Selain itu penulis pernah menjadi asisten dosen di mata kuliah Pemrograman Jaringan, serta asisten praktikum pada mata kuliah Dasar Pemrograman dan Struktur Data.
\end{document}

\end{document} % YAY, WELCOME TO REAL WORD :)
