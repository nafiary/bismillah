\begin{abstrak}
		Semakin berkembangnya teknologi informasi menuntut semakin banyaknya penggunaan perangkat berupa komputer atau perangkat jaringan. Setiap perangkat jaringan memiliki fungsi masing-masing. Sebagai contoh, salah satu fungsi yang dimiliki oleh router adalah untuk medistribusikan alamat kepada tiap host agar tiap host dapat berkomunikasi satu sama lain, lalu ada pula switch yang memiliki fungsi utama yaitu menerima informasi dari berbagai sumber yang tersambung dengannya, kemudian menyalurkan informasi tersebut kepada pihak yang membutuhkannya saja.
		
		Pada suatu organisasi yang besar, kubutuhan akan perangkat jaringan sangatlah besar, terutama dalam hal jumlah perangkat yang digunakan. Banyaknya jumlah perangkat jaringan yang digunakan otomatis membuat jumlah perangkat jaringan yang dipantau juga banyak jumlahnya. Dikarenakan banyaknya jumlah perangkat jaringan yang harus dipantau, seringkali para teknisi mengalami kesulitan dalam memantau kinerja dari tiap perangkat jaringan. Oleh karena itu, dibutuhkan sebuah sistem yang dapat memantau kinerja dari setiap perangkat jaringan yang terpasang.
		
		Aplikasi yang dirancang pada tugas akhir ini, menghadirkan sebuah sistem berbasis web yang dapat memantau seluruh perangkat jaringan yang terhubung dalam jaringan dengan metode publish/subscribe, sehingga setiap user nantinya dapat memilih perangkat jaringan apa saja yang ingin dipantau, dan dapat memilih informasi apa saja yang ingin didapat kan dari tiap-tiap perangkat jaringan yang telah dipilih. \\

	\noindent \textbf{Kata-Kunci}: aplikasi web, publish/subscribe
\end{abstrak}
\newpage
\begin{abstract}
		The development of information technology requires the increasing of devices used in the form of computers or network devices. Every network device has its own function. For example, one of the functions owned by a router is to distribute the address to each host so that the host can communicate with each other, then there is also a switch that has the main function to receiving information from various sources connected to it, then channeling the information to the party who need it only.
		
		In a large organization, the need for network devices is enormous, especially in the number of devices used. The large number of network devices used, automatically keeps the number of network devices being monitored also in increased. Due to the large number of network devices to monitor, it brings technicians into a trouble to monitor the performance of each network device frequently. Therefore, a system that can monitor the performance of each installed network device is required.
		
		The application designed in this final project, presents a web-based system that can monitor all network devices connected in the network with the method of publish / subscribe, so that each user will be able to choose any network device that they want to be monitored, and can choose what information that they want to get from each network device that has been selected. \\

	\noindent \textbf{Keywords}: autoscale, docker, elastic cloud, web application
\end{abstract}