\chapter{PENDAHULUAN}
	Pada bab ini akan dipaparkan mengenai garis besar Tugas Akhir yang meliputi latar belakang, tujuan, rumusan dan batasan permasalahan, metodologi pembuatan Tugas Akhir, dan sistematika penulisan.
        
	\section{Latar Belakang}
		Saat ini, dengan didukung oleh konsep SaaS (\textit{Software as a Service}), aplikasi web berkembang dengan pesat. Banyak perusahaan, seperti Google, Amazon, dan Microsoft yang berhasil mencapai kesuksesan dari aplikasi web. Para penyedia layanan aplikasi web juga berlomba-lomba memberikan pelayanan yang terbaik, seperti menjaga QoS (\textit{Quality of Service}) sesuai dengan perjanjian yang tertuang dalam SLA (\textit{Service Level Agreement}) \cite{kan_docloud:_2016}. Hal tersebut dikarenakan permintaan akses ke suatu aplikasi web biasanya meningkat dengan seiring berjalannya waktu. Keramaian akses sesaat menjadi hal yang umum dalam aplikasi web saat ini. Saat hal tersebut terjadi, aplikasi web akan diakses lebih banyak dari keadaan biasanya. Jika aplikasi web tersebut tidak menyediakan kemampuan untuk menangani hal tersebut, bisa menyebabkan aplikasi web tidak dapat berjalan dengan semestinya yang sangat merugikan pengguna. Biasanya, pengembang akan melakukan pengaturan sumber daya \textit{server} secara manual agar bisa menangani permasalahan di atas, tapi akan memakan banyak biaya dan waktu. Tapi jika tidak ditangani, akibatnya aplikasi web tidak bisa berjalan saat pengalami puncak permintaan dari pengguna. Saat ini banyak tersedia layanan komputasi awan, yaitu sebuah model komputasi yang mana pengguna akan membayar sesuai dengan sumber daya yang digunakan. Dengan bantuan dari komputasi awan, penggembang bisa melakukan \textit{scale up} dan \textit{scale down} sumber daya \textit{server} dari aplikasi web secara manual atau memanfaatkan API yang disediakan oleh \textit{platform} yang bisa diakses dalam rentang waktu jam bahkan menit. Perkembangan dari komputasi awan melahirkan teknologi yang dikenal dengan \textit{autonomos elastic cloud}, sebuah sistem yang secara dinamis akan menambahkan sumber daya sesuai dengan jumlah permintaan. Saat permintaan akses ke suatu aplikasi web meningkat, \textit{elastic cloud} secara otomatis akan menambahkan sumber daya untuk aplikasi dan juga secara otomatis akan mengurangi sumber daya dari aplikasi saat permintaan aksesnya menurun.\\
		\indent \textit{Elastic cloud} merupakan salah satu bagian dari komputasi awan yang sedang populer, dimana banyak riset dan penelitian yang berfokus di bidang ini. Saat ini, biasanya \textit{elastic cloud} berbasis pada \textit{virtual machines} (VMs). VM sendiri dianggap terlalu berat untuk menjalankan sebuah aplikasi web, karena biasanya yang dibutuhakan oleh suatu aplikasi web hanya \textit{web server} (Apache, Nginx), bahasa pemrograman yang digunakan, basis data, dan komponen lainnya, tidak keseluruhan sistem operasi yang terjadi jika menggunakan VM. Dalam hal ini, menggunakan VM untuk mengembangkan aplikasi web hanya akan membuang-buang sumber daya dan menurunkan performa dari aplikasi. Selain itu, penerapan \textit{elastic cloud} yang berjalan di atas VM tidak bisa meningkatkan sumber daya dengan cepat yang bisa merusak QoS.\\
		\indent Sebuah perangkat lunak bernama \textit{docker} dapat menyelesaikan permasalahan dari VM. \textit{Docker} adalah sebuah perangkat lunak yang berfungsi sebagai wadah untuk membungkus dan memasukkan sebuah perangkat lunak ke dalam sebuah lingkungan beserta semua hal yang dibutuhkan oleh perangkat lunak tersebut. Selain membungkus aplikasi, \textit{docker} menjadikan aplikasi yang berjalan di atasnya menjadi terisolasi sehingga menghilangkan kemungkinan terjadinya kebocoran suatu proses aplikasi yang bisa menyebabkan kerusakan pada \textit{host}. \textit{Docker container} berjalan di atas \textit{host} dan menggunakan \textit{kernel} yang sama dengan \textit{host} yang mana memungkinkan \textit{container} dapat dibangun dengan cepat dan membuat penggunaan sumber daya menjadi lebih efisien.\\
		\indent Dalam tugas akhir ini akan dibuat sebuah rancangan sistem yang memungkinkan untuk menjalankan aplikasi web berbasis \textit{docker}. Sistem ini bisa beradaptasi sesuai dengan kebutuhan dari aplikasi yang sedang berjalan. Jika aplikasi membutuhkan sumber daya tambahan, sistem akan menyediakan sumber daya berupa suatu \textit{container} baru secara otomatis dan juga akan mengurangi penggunaan sumber daya jika aplikasi sedang tidak membutuhkannya. Proses skalabilitas ini termasuk skalabilitas secara horizontal, yaitu dengan menambah \textit{instance}, dalam kasus ini berupa \textit{docker container}, dari aplikasi web. Sistem ini juga menyediakan sebuah \textit{server docker repository} untuk menaruh aplikasi web dalam format \textit{docker}. Pengembang yang ingin memasang atau memperbarui aplikasinya di sistem ini akan melakukan \textit{push} aplikasi web dalam format \textit{docker} ke \textit{server repository} dan sistem secara otomatis akan membangun atau memperbarui aplikasi tersebut di \textit{server master host}.

	\section{Rumusan Masalah}
       	Rumusan masalah yang diangkat dalam tugas akhir ini adalah sebagai berikut :
		\begin{enumerate}
			\item Bagaimana cara membuat sistem yang dapat melakukan skalabilitas secara otomatis terhadap aplikasi web berbasis \textit{docker} dengan menggunakan \textit{Proactive Model} dan \textit{Reactive Model}?
			\item Bagaimana cara membuat sistem yang dapat mendistribusikan akses pengguna ke aplikasi web berbasis \textit{docker} secara efisien?
            \item Bagaimana cara membuat sistem yang dapat melakukan pembaruan untuk sebuah aplikasi web yang sudah berjalan tanpa terjadi \textit{downtime}?
		\end{enumerate}

	\section{Batasan Masalah}
		Dari permasalahan yang telah diuraikan di atas, terdapat beberapa batasan masalah pada tugas akhir ini, yaitu:
		\begin{enumerate}
			\item Semua \textit{container} dari aplikasi web akan berjalan hanya pada satu \textit{server master}.
            \item Perhitungan algoritma skala akan menggunakan \textit{Proactive Model} dan \textit{Reactive Model} untuk menentukan jumlah \textit{container} yang dibentuk atau dihapus.
            \item Aplikasi web yang diuji coba hanya akan melakukan komputasi tanpa terhubung dengan layanan luar, seperti koneksi ke suatu basis data dan layanan REST API.
		\end{enumerate}

	\section{Tujuan}
       	Tujuan dari pembuatan tugas akhir ini adalah membuat sistem yang dapat melakukan skalabilitas secara otomatis terhadap aplikasi web berbasis \textit{docker} dengan menggunakan \textit{Proactive Model} dan \textit{Reactive Model} untuk menentukan sumber daya yang diperlukan oleh aplikasi. Selain itu, sistem ini juga memiliki fitur \textit{hot-upgrade}, yaitu dapat melakukan pembaruan terhadap aplikasi yang sudah terpasang tanpa terjadi \textit{downtime}.
        
	\section{Manfaat}
		Tugas akhir ini diharapkan dapat memberikan kemudahan seorang pengembang aplikasi berbasis web dengan tidak perlu melakukan konfigurasi \textit{server} secara langsung untuk melakukan skalabilitas aplikasinya. Pengembang tidak perlu mengawasi aplikasinya saat terjadi perubahan permintaan yang tiba-tiba melonjak tinggi kemudian mengaturnya supaya bisa mengatasi permintaan tersebut. Sistem akan secara otomatis melakukan hal tersebut. Untuk menggunakan sistem ini, pengembang hanya perlu menyimpan aplikasinya di sebuah \textit{server docker repository} dan sistem akan mengelolanya lebih lanjut.
