\chapter{PENDAHULUAN}
	Pada bab ini akan dipaparkan mengenai garis besar Tugas Akhir yang meliputi latar belakang, tujuan, rumusan dan batasan permasalahan, metodologi pembuatan Tugas Akhir, dan sistematika penulisan.
        
	\section{Latar Belakang}
		Semakin berkembangnya teknologi informasi menuntut semakin banyaknya penggunaan perangkat berupa komputer atau perangkat jaringan. Setiap perangkat jaringan memiliki fungsi masing-masing. Sebagai contoh, salah satu fungsi yang dimiliki oleh router adalah untuk medistribusikan alamat kepada tiap host agar tiap host dapat berkomunikasi satu sama lain, lalu ada pula switch yang memiliki fungsi utama yaitu menerima informasi dari berbagai sumber yang tersambung dengannya, kemudian menyalurkan informasi tersebut kepada pihak yang membutuhkannya saja.
		
		Pada suatu organisasi yang besar, kubutuhan akan perangkat jaringan sangatlah besar, terutama dalam hal jumlah perangkat yang digunakan. Banyaknya jumlah perangkat jaringan yang digunakan otomatis membuat jumlah perangkat jaringan yang dipantau juga banyak jumlahnya. Dikarenakan banyaknya jumlah perangkat jaringan yang harus dipantau, seringkali para teknisi mengalami kesulitan dalam memantau kinerja dari tiap perangkat jaringan. Oleh karena itu, dibutuhkan sebuah sistem yang dapat memantau kinerja dari setiap perangkat jaringan yang terpasang.
		
		Aplikasi yang dirancang pada tugas akhir ini, menghadirkan sebuah sistem yang dapat memantau seluruh perangkat jaringan yang terhubung dalam jaringan dengan metode publish/subscribe, sehingga setiap user nantinya dapat memilih perangkat jaringan apa saja yang ingin dipantau, dan dapat memilih informasi apa saja yang ingin didapat kan dari tiap-tiap perangkat jaringan yang telah dipilih.

	\section{Rumusan Masalah}
       	Rumusan masalah yang diangkat dalam tugas akhir ini adalah sebagai berikut :
		\begin{enumerate}
			\item Bagaimana cara membuat agen polling untuk mengambil data pada suatu perangkat jaringan?
			\item Bagimana cara mengimplementasikan publish/subscribe sebagai middleware?
            \item Bagaimana cara menyaring informasi yang dipilih oleh pelanggan?
		\end{enumerate}

	\section{Batasan Masalah}
		Dari permasalahan yang telah diuraikan di atas, terdapat beberapa batasan masalah pada tugas akhir ini, yaitu:
		\begin{enumerate}
			\item User hanya dapat memonitoring perangkat jaringan (server tidak termasuk).
            \item Parameter untuk memonitor perangkat jaringan adalah ketersediaan dan beban yang ditampung.
            \item Performa yang diukur adalah response time.
		\end{enumerate}

	\section{Tujuan}
       	Tugas akhir dibuat dengan beberapa tujuan. Berikut beberapa tujuan dari pembuatan tugas akhir:
       	\begin{enumerate}
       		\item User hanya dapat memonitoring perangkat jaringan (server tidak termasuk).
       		\item Parameter untuk memonitor perangkat jaringan adalah ketersediaan dan beban yang ditampung.
       		\item Performa yang diukur adalah response time.
       	\end{enumerate}
        
	\section{Manfaat}
		Manfaat dari pembuatan tugas akhir ini antara lain adalah:
		\begin{enumerate}
			\item Memonitor ketersediaan dan beban pada sebuah perangkat jaringan.
			\item Memudahkan user untuk memonitoring perangkat jaringan yang diinginkan.
		\end{enumerate}
